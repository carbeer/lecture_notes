\chapter{BWL: Finanzwirtschaft und Rechnungswesen}

Zusammenfassung der Vorlesung "`BWL: Finanzwirtschaft und Rechnungswesen"' aus dem Wintersemester 2014.\footnote{\url{https://ilias.studium.kit.edu/goto_produktiv_crs_369792.html}}



\section{Einführung}

\subsection{Investitionen}
\begin{itemize}
	\item Wertschöpfung: Projekte, deren Ertrag die Kosten der Finanzierung übersteigt
	\item Finanzierungskosten orientieren sich an der Höhe des Risikos der Projekte
	\item Investitionsobjekte: Sachinvestitionen, Immaterielle Investitionen oder Finanzinvestitionen
	\item Investitionsanlass: Gründungs-/Erstinvestitionen, Reininvestitionen oder Erweiterungsinvestitionen
\end{itemize}


\subsection{Wertschöpfung durch geschickte Finanzierung}
\begin{itemize}
	\item Ausnutzung von Fehlinformationen oder irrationalem Investorenverhalten
	\item Ausnutzung von institutionellen Verzerrungen, z.B. der unterschiedlichen Bestuerung der verschiedenen Finanzierungsformen
	\item Vermeidung von riskanten Finanzierungsstrategien
	\item Nutzung von Finanzierungsinstrumenten, die Unternehmen effizienter machen
\end{itemize}


\subsection{Finanzierungsarten}
\begin{itemize}
	\item \textbf{Außenfinanzierung}
	\begin{itemize}
		\item Fremdfinanzierung: Kredite, Anleihen
		\item Beteiligungsfinanzierung: Aktien, Anleihen
	\end{itemize}
	\item \textbf{Innenfinanzierung}
	\begin{itemize}
		\item Selbstfinanzierung: Gewinnrücklagen, Stille Reserven
		\item Sonstige Finanzierung: Rückstellungen
	\end{itemize}
\end{itemize}



\section{Bewertung von Anleihen}

\subsection{Grundlegendes}
Verbriefte, typischerweise handelbare Ansprüche gegenüber einem Schulder (Emittenten).
\begin{itemize}
	\item Dienen der langfristigen Finanzierung
	\item \textbf{Arten}
	\begin{itemize}
		\item Kuponanleihe: Periodische, fixe Zinszahlungen (Kupons) sowie endfällige Tilgungszahlung
		\item Floater: Periodisch variable Zinszahlung, die sich an kurzfristigen Zinsen (z.B. EURIBOR) orientiert mit endfälliger Tilgungszahlung
		\item Nullkuponanleihe (Zerobond): Keine periodischen Zinszahlungen, Tilgungsbeitrag wird am Ende ausgezahlt
		\item Verschiedene Hybride Formen
	\end{itemize}
	\item \textbf{Weitere Gestaltungsmöglichkeiten}
	\begin{itemize}
		\item Sicherheiten durch weitere Vermögensgegestände (z.B. durch Aktienbestände)
		\item Vertragliche Zusatzvereinbarungen (Covenants)
		\item Kündigungsrechte
	\end{itemize}
\end{itemize}


\subsection{Ratings}
\begin{itemize}
	\item Bonitätsbeurteilung des Emittenten oder einer einzelnen Anleihenemission durch unabhängige Agenturen
	\item \textbf{Funktionen}
	\begin{itemize}
		\item Erhöhung des Informationsstands aller Marktteilnehmer
		\item Ermöglichung des Erwerbs der Anleihe durch regulierte Institutionen
	\end{itemize}
\end{itemize}



\section{Methoden der Investitionsentscheidungen}

\subsection{Capital Budgeting}
Der Entscheidungsprozess über die Durchführung von Investitionen.

\subsubsection{Kapitalwert (Net Present Value) einer Investition}

\[Kapitalwert = -Anfangsauszahlung + \frac{Barwert}{1+r}+...+\frac{Barwert}{(1+r)^T}\]

\begin{itemize}
	\item Entscheidungsregel: Führe die Investition durch, wenn Kapitalwert positiv.
	\item Ist das einzusetzende Kapital knapp und infolgedessen können nicht alle Projekte mit positivem Kapitalwert durchgeführt werden, liefert die Kapitalwertmethode keine Lösung des Auswahlproblems.
\end{itemize}

\subsubsection{Methode der Kapitalwertrate}
Lediglich für unabhängige Projekte geeignet.

\[KWR = \frac{Barwert}{Anfangsauszahlung} = \frac{Alle~diskontierten~Auszahlungen}{Anfangszahlung}\]

\subsubsection{Amortisierungsrechnung}
\begin{itemize}
	\item Ziel: Berechnung der Zeitspanne, in der die Anfangsauszahlung wieder in Form kumulierter Zahlungen zurückgeflossen ist.
	\item Entscheidungsregel: Bestimmte, maximale Amortisierungsdauer
	\item \textbf{Vorteile}
	\begin{itemize}
		\item Schnell und einfach anzuwenden
		\item Fehlentscheidungen werden schnell offenbar, z.B. wenn geplante Zahlungsströme nicht eintreten
		\item Unternehmen können zeigen, dass sich Investitionen lohnen
	\end{itemize}
	\item \textbf{Nachteile}
	\begin{itemize}
		\item Keine Berücksichtigung des Zeitpunkts der Zahlungen (später vs. heute)
		\item Keine Berücksichtigung von späteren Zahlungen
		\item Willkürliche Bestimmung der gewünschten Amortisationsdauer
	\end{itemize}
\end{itemize}

\subsubsection{Interne Zinssatzmethode}
\begin{itemize}
	\item Der interne Zinssatz ist derjenige Zinssatz, bei dem der Kapitalwert einer Investition 0 beträgt.
	\item Entscheidungsregel: Investiere, wenn der interne Zinssatz größer als der geforderte Zinssatz ist.
	\item Berechnung: Setze Kapitalwert gleich Null; berechne internen Zinssatz und vergleiche
	\item Vorteile: Simple Kennzahl als Diskussionsgrundlage; Berücksichtigung von mehrperiodigen Zahlungen
	\item \textbf{Nachteile}
	\begin{itemize}
		\item Schwierige Berechnung bei sehr langen Projekten
		\item Bei Vorzeichenwechsel der Zahlungen ergeben sich durch die Potenzen der Zinsen mehrere, mathematisch sinnvolle, Lösungen \(\rightarrow\) bei Vorzeichenwechsel nicht sinnvoll anwendbar
	\end{itemize}
	\item \textbf{Sich ausschließende Projekte}
	\begin{itemize}
		\item Bei Projekten mit unterschliedlichen Größenordnungen nicht sinnvoll anwendbar
		\item Lösung: Betrachtung der Zahlungsströme beider Projekte
	\end{itemize}
\end{itemize}



\section{Bewerten von Aktien}

\subsection{Empirische IPO-Phänomene}
\begin{itemize}
	\item Underpricing: Bookbuilderpreis meist weit unter erstem Handelskurs
	\item Zyklizität: Anzahl der IPOs abhängig vom aktuellen Börsenklima
	\item Kosten: Transaktionskosten relativ hoch
	\item Langfristige Performance: Erst 3-5 Jahre meist eher durchschnittlich
\end{itemize}


\subsection{Kapitalbeschaffung durch Kapitalerhöhung}
\begin{itemize}
	\item Ordentliche Kapitalerhöhung: Ausgabe von jungen Aktien zur Beschaffung von neuem Eigenkapital
	\item Bedingte Kapitalerhöhung: Kapitalerhöhung durch Gebrauch von Umtausch- oder Bezugsrechten
	\item Genehmigtes Kapital: Ermächtigung durch die Hauptversammlung, für befristete Zeit eigenmächtig das Kapital erhöhen zu können
\end{itemize}


\subsection{Dividenden}
\begin{itemize}
	\item Im perfekten Kapitalmarkt sinkt der Aktienkurs um den Betrag der ausgezahlten Dividende (der Gesamtmarktwert sinkt um die Summe der ausgezahlten Dividenden)
	\item Können nicht steuerlich abgesetzt werden (Ausschüttung aus Nachsteuererträgen)
	\item Können nicht zur Insolvenz führen (keine Auszahlungsverpflichtung)
	\item Dividend Smoothing: Langfristig stabile Dividenden, da Unternehmen sich schwer tun, diese zu verändern
	\item \textbf{Signalwirkung}
	\begin{itemize}
		\item Signal des Managements bezüglich der Gewinnerwartung (Dividendenerhöhung/-senkung)
		\item Es gibt allerdings auch Ausnahmen
	\end{itemize}
	\item Dividenden sind meist mit einem höheren Steuersatz belegt als Kapitalgewinne, die Investoren durch den Verkauf gestiegener Aktien verdienen können $\rightarrow$ anpassbar an die Steuerpreferenz der Aktionäre (Klienteleffekt). Trotz steuerlichen Nachteilen ein weiterhin verwendetes Mittel in der Ausschüttungspolitik
\end{itemize}

\subsubsection{Steuerarbitrage}
\begin{itemize}
	\item Empirische Beobachtung: Zunahme des Handelsvolumen bei Ankündigung und um den Ex-Dividenden-Termin herum
	\item Niedrig bestuerte Investoren kaufen Aktien vor dem Ex-Dividenden-Termin und verkaufen sie anschließend wieder
\end{itemize}


\subsection{Bewertung von Aktien}
\begin{itemize}
	\item Aktiengenerierter Zahlungsstrom unsicher $\rightarrow$ Bewertung der Aktie durch Diskontierung des erwarteten Zahlungsstrom
	\item \textbf{Quellen für Zahlungen}
	\begin{itemize}
		\item Dividenden während der Haltedauer
		\item Mittelzufluss durch Aktienverkäufe
	\end{itemize}
\end{itemize}



\section{Portfoliotheorie}
\begin{itemize}
	\item Fundament der modernen Kapitalmarkttheorie: Abwägung zwischen Ertrag und Risiko
	\item Idee: Risikominimierung durch Diversifikation
\end{itemize}


\subsection{Portefeuilles}
Investoren betrachten keine isolierten Wertpapiere sondern Portefeuilles.

\subsubsection{Erreichbare Rendite/Risiko-Kombinationen}
\begin{itemize}
	\item Korrelationskoeffizient hat Einfluss auf Ausmaß der Diversifikationsmöglichkeiten
	\item \(\rho=1 \rightarrow\) Keine Risikorediktion möglich
	\item \(\rho=-1 \rightarrow\) Risiko kann vollständig eleminiert werden
\end{itemize}

\subsubsection{Erreichbare Kombinationen effektiver Portefeuilles}
\begin{itemize}
	\item Berechnung des Globalen Varianzminimalen Portefeilles (GMVP)
	\item Die Porteifeilles oberhalb des GVMP sind effizient
	\item Konkrete Wahl abhängig von der Risikoeinstellung
\end{itemize}

\subsubsection{Erreichbare Kombinationen mit Tangentialportefeuille}
\begin{itemize}
	\item Zusätzliches, risikoles Papier; effiziente Portefeuilles liegen auf einer Geraden
	\item Berechnung des Schnittpunkts der Kurven
	\item Durch Verschuldung mit dem risikolosen Papier sind höhere Investitonen möglich
\end{itemize}


\subsection{CAPM}
\begin{itemize}
	\item Jeder Investor hält Kombination aus Tangentialportefeuille und risikolosem Instrument (Aufteilung investorspezifisch)
	\item Bei aggregierter Nachfrage und festem Angebot: Eventuell Anpassung der Kurse, so dass Markträumung stattfindet
\end{itemize}

\subsubsection{Zerlegung des Gesamtrisikos}
\begin{enumerate}
	\item Risikobeitrag zum Gesamtrisiko: Höhere Rendite \(cov(r_j,r_M)\)
	\item Unternehmensbezogenes Risiko: Keine höhere Rendite \(\sigma^2_{\epsilon_j}\) (höheres Risiko wird nicht durch den Markt vergütet)
\end{enumerate}



\section{Grundlagen des externen Rechnungswesens}
Erfolg resultiert aus der Wertveränderung der Vermögensgegenständen und der Schulden.

\subsection{Erfolgsmessung auf Basis von Marktwerten}
\begin{itemize}
	\item \textbf{Probleme bei der Bestimmung der Marktwerte von Vermögensgegenständen}
	\begin{itemize}
		\item Liquidationswert meinst irrelevant (Vermögengegenstände werden ohnehin nicht verkauft)
		\item Kombinationen von Vermögensgegenständen (nur in diesKombination von Bedeutung)
		\item Subjektivität
		\item Bei unklaren Marktwerten unklar was gebucht werden kann
	\end{itemize}
	\item \textbf{Probleme bei der Bestimmung der Marktwerten von Schulden}
	\begin{itemize}
		\item Marktwerte nur beobachtbar, wenn die Schulden gehandelt werden (Wertpapiere)
		\item Marktwert auch hier meist irrelevant für die Erfolgsermittlung
	\end{itemize}
	\item \textbf{Bewertung der Veränderung des Eigenkapitals}
	\begin{itemize}
		\item Nur möglich, wenn die Unternehmen gehandelt werden
		\item Kalkulierte Werte entsprechen normalerweise nicht dem Börsenwert
		\item Marktteilnehmer übernehmen so die Bewertungsproblematik
		\item Henne-Ei-Problem: Unternehmen wartet auf Bewertung durch die Aktionäre, bzw. umgekehrt
	\end{itemize}
\end{itemize}


\subsection{Grundideen des Rechnungswesens}
\begin{itemize}
	\item Abbildung der ökonomischen Sachverhalte
	\item Regelmäßige Erfassung, Weiterverrechnung und systematische Bewertung
	\item Erträge spiegeln den Wert der erbrachten Güter innerhalb einer Periode wieder
	\item Aufwendungen entsprechen den Werten, die eingesetzt wurden, um den Ertrag zu erzielen
	\item Grundprinzipien: Verständlichkeit, Entscheidungsrelevanz, Verlässlichkeit, Vergleichbarkeit
\end{itemize}


\subsection{Prinzipien}
\begin{itemize}
	\item Realisationsprinzip: Umsätze gelten als realisiert, wenn die geschäftliche Transaktion komplett abgeschlossen ist
	\item Matching Principle: Die Umsätze einer Periode werden vor den Aufwendungen einer Periode verbucht (z.B. erst wenn ein Produkt verkauft wurde)
	\item Prudence Principle: Potentielle Verluste werden so früh wie möglich ausgewiesen, Gewinne erst, wenn sie realisiert worden sind
	\item Going Concern Principle: Es wird davon ausgegangen, dass das Unternehmen in unabsehbarer Zukunft weiter aktiv bleibt
\end{itemize}


\subsection{Die Bilanz}
\begin{itemize}
	\item \textbf{Aktiva: Mittelverwendung}
	\begin{itemize}
		\item Anlagevermögen (langfristig)
		\begin{itemize}
			\item Sachanlagen (Gebäude, Fuhrpark, Maschinen)
			\item Finanzanlagen (Aktien, Staatsanleihen, Beteiligungen)
			\item Immaterielle Vermögengegenstände (Lizenzen, Patente, Technologien)
			\item Aktive latente Steuern
		\end{itemize}
		\item Umlaufvermögen (kurzfristig)
		\begin{itemize}
			\item Liquide Mittel
			\item Vorräte
			\item Forderungen
			\item Geleistete Anzahlungen
		\end{itemize}
	\end{itemize}
	\item \textbf{Passiva: Mittelherkunft}
	\begin{itemize}
		\item Eigenkapital
		\begin{itemize}
			\item Reinvermögen des Unternehmens
			\item Unternehmenswert aus der Perspektive der Eigentümer
			\item Bestandteile
			\begin{itemize}
				\item Gezeichnetes Kapital
				\item Kapitalrücklagen (eingezahlt von den Gesellschaftern)
				\item Gewinnrücklagen (kumuliert aus bisherigen Gewinnen)
			\end{itemize}
		\end{itemize}
		\item Fremdkapital
		\begin{itemize}
			\item Finanzielle Verbindlichkeiten (Überziehungskredite, Bankdarlehen, Hypotheken, Anleihen)
			\item Verbindlichekeiten aus operativem Geschäft
			\item Erhaltene Anzahlungen
			\item Rückstellungen
			\item Passive latente Steuern
		\end{itemize}
	\end{itemize}
\end{itemize}


\subsection{Gewinn- und Verlustrechnung}
\begin{itemize}
	\item Wertschaftlicher Erfolg einer Periode
	\item \(Umsatzerloese - Herstellungskosten = Bruttoergebnis\)
	\item \(Bruttoergebnis - Periodenaufwendungen = Operativer Gewinn/Verlust\)
	\item \(Operatives Ergebnis - Kosten Fremdkapital = Ergebnis vor Steuern\)
	\item \(Ergebnis vor Steuern - Steuern auf Gewinn = Ergebnis nach Steuern\)
\end{itemize}


\subsection{Worksheet-Approach}
Tabellarische Darstellung der Geschäftsvorfälle.

\subsubsection{Zahlungsströme aus Finanzierungstätigkeit}
\begin{itemize}
	\item Kreditaufnahme: Liquide Mittel \((+)\) \(\rightarrow\) Fremdkapital \((+)\)
	\item Kredittilgung: Liquide Mittel \((-)\) \(\rightarrow\) Fremdkapital \((-)\)
	\item Zinszahlung: Liquide Mittel \((-)\) \(\rightarrow\) Eigenkapital \((-)\)
	\item Kapitalerhöhung: Liquide Mittel \((+)\) \(\rightarrow\) Eigenkapital \((+)\)
	\item Aktienrückkauf: Liquide Mittel \((-)\) \(\rightarrow\) Eigenkapital \((-)\)
	\item Dividendenzahlung: Liquide Mittel \((-)\) \(\rightarrow\) Eigenkapital \((-)\)
\end{itemize}

\subsubsection{Rechnungsabgrenzung}
\begin{itemize}
	\item Antizipative Rechnungsabgrenzungsposten: Vorwegnahme von Aufwendungen oder Umsatzerlösen in dieser Periode
	\item Transitorische Rechnungsabgrenzungsposten: Zahlungen bereits in dieser Periode, Aufwendungen oder Umsatzerlöse in der nächsten
\end{itemize}

\subsubsection{Beispiele}
Siehe zusätzlich "`Zahlungsströme aus Finanzierungstätigkeit"'
\begin{itemize}
	\item Bewertung Warenbestand Autos: Einkaufswert + Reparaturkosten
	\item Eigenkapital bei Bilanzeröffnung: Aktiva und Passiva ausgleichen
	\item Einzahlung: Liquide Mittel \((+)\) \(\rightarrow\) Eigenkapital \((+)\)
	\item Kauf von Waren: Liquide Mittel \((-)\) \(\rightarrow\) Warenbestand \((+)\)
	\item \textbf{Verkauf von Waren}
	\begin{itemize}
		\item GuV: Umsatzerlöse \((+)\) \(\rightarrow\) Herstellungskosten \((-)\)
		\item Die Differenz davon ergibt sich als Gewinn oder Verlust im Eigenkapital
	\end{itemize}
	\item Aufbereitung von Waren (z.B. Reinigung): Liquide Mittel \((-)\) \(\rightarrow\) (Reinigungs-)Kosten \((+)\)
	\item Einkauf von Waren \textit{auf Ziel}: Zahlung erst in 30 Tagen fällig, der Warenbestand wird sofort erhöht
	\item Lieferantenkredit bei Warenverkauf: Warenbestand wird sofort reduziert, die Zahlung wird erst in 30 Tagen fällig
	\item \textbf{Vorauszahlung der Miete für sechs Monate}
	\begin{itemize}
		\item Liquide Mittel \((-)\) \(\rightarrow\) Geleistete Anzahlungen \((+)\)
		\item Die monatliche Miete: Geleistete Anzahlungen \((-)\) \(\rightarrow\) Mietaufwand \((+)\) und somit negativ im Eigenkapital
	\end{itemize}
	\item Operatives Ergebnis = Erträge gesamt - Aufwendungen gesamt
\end{itemize}



\section{Methodik des externen Rechnungswesens}

\subsection{Abschreibungen}
\begin{itemize}
	\item Verteilung des Gesamtaufwands über mehrere Perioden
	\item Restwert soll potentiellem Verkaufswert entsprechen
	\item Verschiedene Abschreibungsmethoden: Linear, degressiv, progressiv
	\item Auch selbst produzierte Vermögensgegenstände können aktiviert und abgeschrieben werden
\end{itemize}

\subsubsection{Außerplanmäßige Abschreibungen}
\begin{itemize}
	\item Beispiele: Katastrophen, Fehlinvestitionen, Bonitätsverlust, Kursrückgang, Preissenkung
	\item Liegt der Buchwert unter dem heutigen Marktwert und dem indirekten Wert, wird außerplanmäßig zum jeweils höcgsten Wert abgeschrieben
\end{itemize}


\subsection{Vorratsbewertung}
\begin{itemize}
	\item Bewertung jedes Vorratsgegenstands in der Bilanz oft unmöglich oder unverhältnismäßig aufwendig
	\item \textbf{Bewertungsmethoden}
	\begin{itemize}
		\item Gleitender Durchschnitt: Durchschnitt aus Anfangsvorrat und Einkäufen
		\item FIFO: Die ersten hinzugefügten Kosten werden kalkulatorisch zuerst entnommen
		\item LIFO: Die letzten Kosten werden kalkulatorisch zuerst entnommen
	\end{itemize}
\end{itemize}


\subsection{Rückstellungen}
\begin{itemize}
	\item Passivierte Verpflichtungen zu zukünftigen Auszahlungen \(\rightarrow\) Schulden
	\item Beispiele: Pensionsrückstellungen, Gewährleistungen, ungwisse Verbindlichkeiten (für Prozesse oder Steuern)
\end{itemize}


\section{Grundlagen des internen Rechnungswesens}

\subsection{Definitionen}
\begin{itemize}
	\item Bruttoergebnis = Umsatzerlöse - Umsatzkosten
	\item Operativer Gewinn = Bruttoergebnis - Periodenaufwand
	\item Kostenobjekt: Abstraktes Bezugsobjekt, das Kosten verursacht (Bezugsobjekt)
	\item Kostentreiber: Kosteneinflussfaktor; erklärt, wie die gesamten Kosten eines Kostenobjekts zustande kommen. Bsp: Die Anzahl der Produktvarienten erklärt die Kosten für Maschinen (wegen der Komplexität)
	\item Variable Kosten: Veränderliche, bewegliche, mengenabhängige Kosten. Variieren mit der Ausprägung eines Kostentreibers
	\item Fixe Kosten: Unveränderliche, feste Kosten, die über einen längeren Zeitraum konstant bleiben
\end{itemize}


\subsection{Kostenrechnung}
\begin{itemize}
	\item \textbf{Teilkostenrechnung}
	\begin{itemize}
		\item Variable Kosten fließen in die Umsatzkosten
		\item Fixe Herstllkosten und Nicht-Herstellkosten werden als Periodenaufwand verbucht
	\end{itemize}
	\item \textbf{Vollkostenrechnung}
	\begin{itemize}
		\item Variable und fixe Herstellkosten werden als Umsatzkosten berücksichtigt
		\item Nicht-Herstellkosten sind Periodenaufwand
	\end{itemize}
	\item Wechsel von Teil- zur Vollkostenrechnung: Höhere Umsatzkosten, die aktiviert werden können
	\item Diese werden erst als Aufwand gesehen, wenn die Produkte verkauft werden
\end{itemize}


\subsection{Berechnungsschemata}
\begin{table}[h]
\begin{tabularx}{\textwidth}{|l|X|X|}
	\hline
	& \textbf{Vollkostenrechnung} & \textbf{Teilkostenrechnung} \\
	\hline\hline
	\textbf{Umsatzerlöse} & Summe verkaufter Waren & Summe verkaufter Waren \\
	\hline
	\textbf{Umsatzkosten} & Summe variabler Herstellkosten, Summe fixer Herstellkosten anteilig an Gesamtproduktion & Summe variabler Herstellkosten \\
	\hline
	\textbf{Periodenaufwand} & Nicht-Herstellkosten & Fixe Herstellkosten, Nicht-Herstellkosten \\
	\hline
\end{tabularx}
\end{table}

Die enstandene Differenz spiegelt sich in dem höheren Wert der fertigen Erzeugnisse bei der Vollkostenrechnung wider, geht aber in der Teilkostenrechnung sofort als Periodenaufwand in die Gewinnberechnung ein.

\subsection{Einzel- vs. Gemeinkosten}
\begin{itemize}
	\item Einzelkosten: Werden von Einzelobjekt verursacht und können diesem in exakter Höhe zugeordnet werden. Bsp: Materialkosten
	\item Gemeinkosten: Kosten können NICHT jeweils exakt einzelnen Objekten zugeordnet werden. Zuteilung erfolgt über Schlüssel. Bsp: Stromkosten, Miete
\end{itemize}

\subsubsection{Beispiel: Automobilproduktion}
\begin{table}[h]
\begin{tabularx}{\textwidth}{|l|X|X|}
	\hline
	& \textbf{Einzelkosten} & \textbf{Gemeinkosten} \\
	\hline\hline
	\textbf{Variable Kosten} & Materialkosten der Reifen & Stromkosten der Produktionshalle \\
	\hline
	\textbf{Fixe Kosten} & Gehälter der unmittelbar beteiligten Mitarbeiter & Jährliche Mietkosten der Produktionshalle \\
	\hline
\end{tabularx}
\end{table}


\subsection{Zuordnungsproblematik}
\begin{itemize}
	\item Zuordnung auf Kostenstellen oder Kostenarten nicht immer eindeutig
	\item Prinzipien: Verursachungsprinzip/Pauschalprinzip; Durchschnittsprinzip
	\item Grundsätzlich muss der Grundsatz der Wirtschaftlichkeit beachtet werden
\end{itemize}



\section{Kostenartenrechnung}
\begin{itemize}
	\item Wiederbeschaffungswert: Kosten, ein Produkt zu verkaufen (Umsatzkosten oder Herstellkosten)
\end{itemize}



\section{Kostenstellenrechnung}
\begin{itemize}
	\item Abrechnungseinheiten, in die das Unternehmen aufgegliedert ist
	\item \textbf{Differenzierung}
	\begin{itemize}
		\item Vorkostenstelle: Arbeiten nicht direkt am Endprodukt (sekundäre Kostenstelle)
		\item Endkostenstelle: Kann direkt dem Kostenträger zugeordnet werden (primäre Kostenstelle)
	\end{itemize}
	\item \textbf{Aufgaben}
	\begin{itemize}
		\item Bindeglied zwischen Kostenarten- und Kostenträgerrechnungm, um Gemeinkosten bei Mehrproduktunternehmen auf die Kostenträger verrechnen zu können
		\item Bereitstellung von Informationen über den Leistungsbeziehungs und Ressourcenverbrauch
	\end{itemize}
\end{itemize}


\subsection{Betriebsabrechnungsbogen}
\begin{itemize}
	\item Tabellarische Aufstellung der Kostenstellen
	\item \textbf{Vorgehen}
	\begin{itemize}
		\item Aufgliederung der primären Gemeinkosten auf Kostenstellen (Primrkostenverrechnung)
		\item Verrechnung der innerbetrieblichen Leistungen von Vorkostenstellen auf Endkostenstellen (Sekundärkostenverrechnung)
	\end{itemize}
	\item Sekundärkostenverrechnung: Die Beträge der Vorkostenstellen werden anteilig den Endkostenstellen zugeordnet
	\item Verwendung von kostenstellenspezifischen Kalkulationssätzen (Kostenschlüssel): Bsp. Mengenschlüssel oder Wertschlüssel
\end{itemize}

\subsubsection{Verrechnungsverfahren der Kostenstellen}
\begin{itemize}
	\item \textbf{Anbauverfahren}
	\begin{itemize}
		\item Verrechnungspreise der Kosten der Vorkostenstelle pro abgegebener Leistungseinheit
		\item Beispiel Kantine: Kosten pro Essen ermitteln und danach mit der Anzahl der bezogenen Einheiten der Endkostenstelle verrechnen und summieren
		\item Keine Berücksichtigung der Anteile der anderen VOrkostenstellen
	\end{itemize}
	\item Stufenleiterverfahren: Wie beim Anbauverfahren, allerdings werden die Beträge \textbf{allen} Folgekostenstellen zugeordnet (nicht nur den Endkostenstellen)
	\item \textbf{Gleichungsverfahren}
	\begin{itemize}
		\item Verrechnung der Kosten zwischen allen (beteiligten) Kostenstellen
		\item Kostenentlastung der leistenden Kostenstelle \(\rightarrow\) Kostenbelastung der empfangenden Kostenstellen
	\end{itemize}
\end{itemize}



\section{Kostenträgerrechnung}
\begin{itemize}
	\item Kostenträger sind einzelne Produkte oder Leistungen (Kostenträgerstückrechnung), aber auch Perioden (Kostenträgerzeitrechnung)
	\item \textbf{Aufgaben der Kostenträgerstückrechnung}
	\begin{itemize}
		\item Ermittlung der Herstellkosten oder Selbstkosten
		\item Vorbereitung kurzfristiger Ergebnisrechnung
		\item Berwertung der Bestände an unfertigen und fertigen Erzeugnissen
	\end{itemize}
\end{itemize}


\subsection{Verfahren der Kostenträgerstückrechnung}
Ausführliche Beispielrechnungen in Vorlesung 13.

\subsubsection{Divisionskalkulation}
\begin{itemize}
	\item Durchschnittsbetrachtung als Grundlage: \(Selbstkosten~pro~Stueck=\frac{Gesamtkosten}{Produktionsmenge}\)
	\item Stufenweise: Neue Stufe hat Ergebnis der Vorstufe als Eingabe
	\item Kostenverteilung nach Verursachungsprinzip
	\item Gesamtkosten bilden die Summe der einzelnen Stufen
	\item Falls die fertigen Erzeugnisse gelagert werden, zusätzlich \(\frac{Verwaltungs-~und~Vertriebskosten}{Absatzmenge}\)
\end{itemize}

\subsubsection{Zuschlagskalkulation}
\begin{itemize}
	\item Geeignet Unternehmen mit mehreren Produkten und komplexen Prozessen (z.B. Serienfertigung oder Einzelfertigung)
	\item Idee: Gemeinkosten werden durch Zuschlagssätze entsprechend auf Einzelkosten umgelegt
	\item Differenzierung nach Eintel-, Gemein- und Sondereinzelkosten unter Verwendung des BAB
	\item \textbf{Schema}
	\begin{enumerate}
		\item \(Materialkosten = Fertigungsmaterial + Material-Gemeinkosten-Zuschlag\)
		\item \(Fertigungskosten = Fertigungslöhne + Fertigungs-Gemeinkosten-Zuschlag\)
		\item \(Herstellkosten = Sondereinzelkosten~der~Fertigung\)
		\item \(Selbstkosten = Forschung~und~Entwicklung+Verwaltung + Vertrieb + Sonderkosten~des~Vertriebs\)
	\end{enumerate}
	\item Selbstkosten sind die Summe daraus
\end{itemize}

\subsubsection{Äquivalenzkalkulation}
\begin{itemize}
	\item Idee: Ähnliche Produkte werden in ein einheitliches "`Einheitsprodukt"' umgerechnet. Bsp: verschiedene Biersorten
	\item \textbf{Prozess}
	\begin{enumerate}
		\item Bestimmung der Äquivalenzziffern: Festlegung eines Produkts als Referenz (\(1,0)\), Schätzung der komplexität der anderen
		\item Umrechnung der Outputmengen auf Einheitsmengen \(Einheitsmenge = Anzahl \cdot Aehnlichkeit\)
		\item Ermittlung der Stückselbstkosten: \(\frac{Selbstkosten~Unternehmen}{Einheitsgesamtmenge}\)
		\item Ermittlung der Stückkosten der äquivalenten Produkte
	\end{enumerate}
\end{itemize}

\subsubsection{Maschinenstundensatzkalkulation}
\begin{itemize}
	\item Idee: Differenzierung zwischen maschinenzurechenbaren Gemeinkosten und den nicht-maschinenzurechenbaren Rest-Gemeinkosten
	\item Maschinenzurechenbare Gemeinkosten:\newline \(Maschinenstundensatz = \frac{Maschinenzurechenbare~Gemeinkosten}{Tatsaechliche~Maschinenlaufzeit}\)
	\item Rest-Gemeinkosten: Verrechnung entsprechend Zuschlagskalkulation
	\item \textbf{Generelles Kalkulationsschema}: Summierung von
	\begin{itemize}
		\item Fertigungskosten
		\item Maschinenabhängige Fertigungsgemeinkosten
		\item Restfertigungsgemeinkosten
	\end{itemize}
\end{itemize}



\section{Kostenanalyse}

\subsection{Plankostenrechnung}
\begin{itemize}
	\item Ergebniskontrolle: Kostenrechnerische Abweichungsanalyse
	\item Basis: Planung von Kosten, beispielsweise einer Kostenstelle
	\item \textbf{Abweichungsarten}
	\begin{itemize}
		\item Beschäftigungsabweichung (BA)
		\begin{itemize}
			\item Berechnung: Soll-Kosten - verrechnete Plankosten
			\item Gründe: Nachfrageschwund führt zu falsch dimensionierten Kapazitäten; geringe Vertriebsanstrengung
			\item Verantwortlichkeit: Geschäftsführung
		\end{itemize}
		\item Verbrauchsabweichung (VA)
		\begin{itemize}
			\item Berechnung: Ist-Kosten - Soll-Kosten (der Teil, der sich auf den Verbrauch der Inputfaktoren bezieht)
			\item Grpünde: Ineffiziens; erhöhter Ausschuss in Produktion oder aufgrund schlechter Qualität der Inputfaktoren
			\item Verantwortlich: Kostenstellenleiter/Kostenstellenleiter/Beschaffungsabteilung
		\end{itemize}
		\item Preisabweichung (PA)
		\begin{itemize}
			\item Berechnung: Ist-Kosten - Soll-Kosten (der Teil, der sich auf die Preise der Inputfaktoren bezieht)
			\item Gründe: Exogene Preiserhöhung
			\item Verantwortlich: Beschaffungsabteilung
		\end{itemize}
	\end{itemize}
\end{itemize}


\subsection{Break-Even-Analyse}
\begin{itemize}
	\item Dient der Analyse des Zusammenhangs zwischen der Beschäftigung und dem Gewinn sowie dessen Bestimmungsfaktoren Umsatz und Kosten
	\item Ab welcher Menge (Break-Even-Menge) einer Produktart wird ein positiver Gewinn erzielt
\end{itemize}



\section{Appendix A: Formelsammlung}

\subsection{Anleihen}

\subsubsection{Barwert einer ewigen Rente}
\[BW = \frac{C}{r}\]

\subsubsection{Barwert einer endlichen Rente}
\[BW = \frac{C}{(1+r)^1}+..+\frac{C}{(1+r)^T} = \frac{C}{r}\cdot[1-\frac{1}{(1+r)^n}]\]

\subsubsection{Ewige Rente mit konstantem Wachstum g}
\[BW = \frac{C}{r-g}\]

\subsubsection{Endliche Rente mit konstantem Wachstum}
\[BW = \frac{C}{r-g} \cdot [1-[\frac{1+g}{1+r}]^T]\]

\subsubsection{Yield to Maturity von Zerobonds}
\[y = (\frac{Nennwert}{P_Z})^{\frac{1}{T}})-1\]

\subsubsection{Unterjährige Verzinsung}
\[EW=C_0\cdot[1+\frac{r}{m}]^{mT}\]

\subsubsection{Stetige Verzinsung}
\[EW=C_0\cdot e^{rT}\]

\subsubsection{Investor mit Einjahreshorizont}
\[r = \frac{Div_1+P_1}{P_2}-1\]

\subsubsection{Methoden der Investitionsentscheidung: Interne Zinssatzmethode}
\[0 = Anfangsauszahlung - \sum_{t=1}^{T}\frac{C_t}{(1+z_intern)^t}\]


\subsection{Aktien}

\subsubsection{Aktienrendite}
\[r = r_{Dividende}-r_{Kursveraenderung} = \frac{Div_1}{P_0}-\frac{P_1-P_0}{P_0} = \frac{Div_1+P_1}{P_0} - 1\]

\subsubsection{Aktienbewertung: Dividend Discount Model}
Der Preis der Aktie ergibt sich als Barwert alles erwarteten Dividendenzahlungen bei einem risikoadäquaten Diskontierungssatz \(r_E\).
\[P_0 = \sum_{t=1}^T \frac{Div_t}{(1+r_E)^t}\]

\subsubsection{Divided Discount Model mit unterschiedlichen Wachstumsraten}
Erweiterung um ewige Rente.
\[P_0 = \sum_{t=1}^{T} \frac{Div_t}{(1+r_E)^t} + \frac{1}{(1+r_E)^T}\cdot\frac{Div_{T+1}}{r_E-g}\]


\subsection{Portfoliotheorie}

\subsubsection{Dividendenrendite}
\[Dividendenrendite = \frac{Div}{Kurs}\]

\subsubsection{Rendite aus Kursveränderung}
\[Rendite~aus~Kursveraenderung = \frac{P_1-P_0}{P_0}\]

\subsubsection{Aktienrendite}
Bei verschiedenen Papieren mit linearer Gewichtung.

\[r = \frac{Div + K_1 - K_0}{K_0}\]

\subsubsection{Standardabweichungen der Rendite}
Bei verschiedenen Papieren mit linearer Gewichtung unter der Wurzel.
\[R = \frac{R_1+...+R_T}{T}\]
\[Std = \sqrt{Var} = \sqrt{\frac{(R_1-R)^2+..+(R_T-R)^2}{T-1}}\]

\subsubsection{Erwartungswert der Portefeuillerendite}
\[\mu_w = w_1\mu_1 + w_2\mu_2\]

\subsubsection{Risiko der Portefeuillerendite}
\[\sigma^2_w = w^2_1\sigma^2_1+w^2_2\sigma^2_2+2w_1w_2cov(r_1,r_2)\]

\[cov(r_1,r_2) = \sigma_1\sigma_2\rho_{12} = E((r_1-\mu_1)(r_2-\mu_2))\]

\subsubsection{CAPM: Kapitalmarktlinie}
\[\mu_W = r + \frac{\mu_M-r}{\sigma_M}\cdot\sigma_W\]

\subsubsection{CAPM: Wertpapiermarktlinie}
\[\mu_j = r + \frac{\mu_M-r}{cov(r_M,r_M)}\cdot cov(r_j,r_M) = r + (\mu_M-r)\cdot \beta_j\]
\[\beta_j = \frac{cov(r_j,r_M)}{cov(r_M,r_M)}\]

\subsubsection{Marktrisikoprämie}
\[\mu_M - r\]

\subsubsection{Ausschüttungsquote}
\[\frac{Div~pro~Aktie}{Gewinn~pro~Aktie}\]

\subsubsection{Marktkapitalisierung}
\[Preis~pro~Aktie \cdot n_{Aktie}\]


\subsection{Methodik des externen Rechnungswesens}

\subsubsection{Geometrisch-degressive Abschreibung}
\[p = 1 - \sqrt[T]{\frac{Restwert}{Anfangswert}}\]


\subsection{Verfahren der Kostenträgerstückrechnung}

\subsubsection{Divisionskalkulation}
\[Herstellkosten_{Stufe_i} = \frac{Gesamtkosten_{Stufe_i}}{Outputmenge_{Stufe_i}}\]
\[Einsatzfaktor_{Stufe_i}=\frac{Inputmenge}{Outputmenge}\]
\[Korrekturterm_{Stufe_i} = (Einsatzfaktor_{Stufe_i}-1)\cdot Herstellkosten_{Stufe_i-1}\]
\[k_{Stufe_i} = Herstellkosten_{Stufe_i}+Korrekturterm_{Stufe_i}\]


\subsection{Plankostenrechnung}
\begin{itemize}
	\item Plankosten: \(K_p\)
	\item Fixe Plankosten: \(K_{fp}\)
	\item Ist-Kosten: \(K_i\)
	\item Ist-Beschäftigung: \(x_i\)
	\item Plan-Beschäftigung: \(x_p\)
	\item Konstanter variabler Plankostensatz pro Beschäftigungseinheit: \(KS_{vp}=2,7\)
\end{itemize}

\subsubsection{Planverrechnungssatz pro Beschäftigungseinheit}
\[KS_{verr} = \frac{K_p}{x_p}\]

\subsubsection{Verrechnete Plankosten}
\[K_{verr} = KS_{verr} \cdot x_i\]

\subsubsection{Soll-Kosten}
\[K_s = KS_{vp} \cdot x_i + K_p\]

\subsubsection{Beschäftigungsabweichung}
\[BA = K_s - K_{verr}\]

\subsubsection{Preis-/Verbrauchsabweichung}
\[PAVA = K_i - K_s\]
