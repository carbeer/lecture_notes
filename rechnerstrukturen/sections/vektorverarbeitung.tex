\chapter{Vektorverarbeitung}

\section{Einführung}
\begin{itemize}
	\item Motivation: Gerade in HPC-Anwendungen fallen oft viele gleichartige Daten an, die auf ähnliche Weise verarbeitet werden sollen, so zum Beispiel bei Simulationen in der Meteorologie und Geologie (Single Instruction Multiple Data)\footnote{\url{https://de.wikipedia.org/wiki/Vektorprozessor\#Funktionsweise_und_Anwendungsfelder}}
	\item Verarbeitungsbeispiel: \(Y = a \cdot X + Y\). \(X\) und \(Y\) sind zwei Vektoren gleicher Länge und a ist eine skalare Größe. Dieses Problem wird auf Skalarprozessoren durch eine Schleife gelöst\footnote{\url{https://de.wikipedia.org/wiki/Vektorprozessor\#MIPS-Architekturbeispiel}}, dadurch hoher Aufwand durch viele Schleifendurchläufe, Pipeline-Konflikte oder Mehrzyklusoperationen
	\item Verarbeitung in einem Rechenwerk mit pipeline-artig aufgebauten Funktionseinheiten
	\item \textbf{Code-Beispiel\footnote{\url{https://de.wikipedia.org/wiki/Vektorprozessor\#MIPS-Architekturbeispiel}}: Berechnung von \(Y = a \cdot X + Y\)}\\\\
	\begin{minipage}{\linewidth}
	\begin{lstlisting}[frame=single,numbers=left,mathescape,language={[mips]Assembler},tabsize=4]
; MIPS
		L.D 	 F0, a          ; Skalar a laden
		DADDIU	 R4, Rx, #512   ; letzte Adresse 512/8 = 64
Loop:	L.D 	 F2, 0(Rx)  	; X(i) laden
		MUL.D    F2, F2, F0     ; a * X(i)
		L.D      F4, 0(Ry)      ; Y(i) laden
		ADD.D    F4, F4, F2     ; a * X(i) + Y(i)
		S.D      0(Ry), F4      ; Y(i) speichern
		DADDIU   Rx, Rx, #8     ; Index (i) von X inkrementieren
		DADDIU   Ry, Ry, #8     ; Index (i) von Y inkrementieren
		DSUBU    R20, R4, Rx    ; Rand berechnen
		BNEZ     R20, Loop      ; wenn 0, dann fertig

; VMIPS
		L.D      F0, a       	; Skalar a laden
		LV       V1, Rx      	; Vector X laden
		MULVS.D  V2, V1, F0  	; Vector-Skalar Multiplikation
		LV       V3, Ry      	; Vector Y laden
		ADDV.D   V4, V2, V3  	; Vektor Addition
		SV       Ry, V4      	; Resultat speichern
	\end{lstlisting}
	\end{minipage}
	\item \textbf{Aufbau eines Vektorprozessors}
	\begin{itemize}
		\item Einheit um Vektoren direkt aus Hauptspeicher zu laden
		\item Vektoreinheit: Ein Satz Vektorregister
		\item Mindestens eine Skalareinheit zur Ausführung von Befehlen, die nicht auf ganze Vektoren angewendet werden sollen (Skalarbefehle)
		\item Vektoreinheiten und Skalareinheiten können parallel arbeiten \(\rightarrow\) Vektorbefehle und Skalarbefehle können parallel ausgeführt werden
	\end{itemize}
	\item \textbf{Pipeling eines Vektorprozessors}
	\begin{itemize}
		\item Berechnung eines Ergebnisses pro Takt bei ununterbrochener Arbeit und einer Füllzeit/Einschwingzeit
		\item Beispiel einer Gleitkommaoperation
		\begin{enumerate}
			\item Laden eines Paars von Gleitkommazahlen aus Vektorregister
			\item Vergleich der Exponenten und Verschieben einer Mantisse
			\item Addition der ausgerichteten Mantisse
			\item Normalisieren des Ergebnisses und Schreiben in Zielregister
		\end{enumerate}
		\item Verkettung von Pipelines: Erweiterung auf eine Folge von Vektoroperationen durch Verkettung der (spezialisierten) Pipelines. So können die Ergebnisse einer Pipline sofort der nächsten Pipeline zur Verfügung gestellt werden. Beispielsweise können Addition, Schieben und UND-Verknüofung unmittelbar hintereinander ausgeführt werden
		\item Multifunktions- oder spezialisierte Pipelines zur Realisierung arithemtisch-logischer Verknüpfungen
		\begin{itemize}
			\item Verwendung eines Multifunktionspipeline oder eine Anzahl von spezialisierten Pipelines
			\item Multifunkspipeline: Der Aufbau erfordert eine höhere Stufenzahl, wobei aktuell nicht benötigte Stufen oder Pipelines übersprungen werden können
			\item Spezialisierte Pipelines: Jeweils zur Durchführungs spezieller Funktionen benutzt. Hardware und Steuerung relativ einfach, allerdings werden mehrere unabhängige Pipelibes benötigt, um alle Funktionen abzubilden
		\end{itemize} 
	\end{itemize}
	\item \textbf{Parallelität in einem Vektorrechner}
	\begin{itemize}
		\item Vektor-Pipeline-Parllelität durch die Stufenzahl der Vektor-Pipeline
		\item Mehrere Vektor-Pipelines in einer Vektoreinheit
		\item Vervielfachen der Pipelines: Pro Takt wird nicht nur ein Operandenpaar sondern ein Vielfaches davon verarbeitet
		\item Mehrere Vektoreinheiten, die parallel zueinander arbeiten (vgl. speichergekoppelter Multiprozessor)
	\end{itemize}
	\item \textbf{Parallelarbeit in der Software}
	\begin{itemize}
		\item Vektorisierung der innersten Schleife mittels vektorisierendem Compiler
		\item Vektorverbundbefehle zur Verkettung von Vektorbefehlen (beispielsweise Vector-Multiply-Add)
		\item Die Verteilung einer Berechnung auf mehrere, gleiche Pipelines geschieht durch den Compiler (Vektorisierung der innersten Schleife)
	\end{itemize}
	\item \textbf{Memory Interleaving (Speicherverschränkung)}
	\begin{itemize}
		\item Anpassen der Zugriffsgeschwindigkeit an die Verarbeitungsgeschwindigkeit der CPU
		\item Der Speicher wird in gleich große, voneinander unabhängige Bereiche (Module, Speicherbänke) unterteilt, die zeitlich verschränkt gelesen oder beschrieben werden können. Aufeinander folgende Speicherworte werden zyklisch in aufeinander folgenden Speicherbänken abgespeichert\footnote{\url{https://de.wikipedia.org/wiki/Speicherverschränkung}}
		\item Hat der Speicher eine geringere Taktrate als der Prozessor, verringern sich durch den abwechselnden Zugriff die Wartezeiten für Speicheroperationen \(\rightarrow\) mehr Zeit für langsamere Speicherbausteine
	\end{itemize}
\end{itemize}


\section{Vektorverarbeitung}

\subsection{Eigenschaften}
\begin{itemize}
	\item \textbf{Vektor Stride}
	\begin{itemize}
		\item Problem: Die Elemente eines Vektors liegen nicht immer so wie sie gebraucht werden in aufeinander folgend Speicherzellen. Beispielsweise ändert sich die Reihenfolge bei Matrizenmultiplikationen (Spaltenelemente \(\times\) Zeilenelemente)
		\item Stride-Wert bezeichnet den Abstand zwischen den Elementen. Dieser kann sich allerdings zur Laufzeit ändern oder ist dann erst bekannt. Lösung: Ablegen in Allzweckregister
		\item In heutigen Vektorrechnern werden die Zugriffe von jedem Prozessor auf über mehrere Hundert Speicherbänke verteilt 
	\end{itemize}
	\item \textbf{Bedingte Ausführung}
	\begin{itemize}
		\item Problem: Programme mit if-Anweisungen in Schleifen können nicht vektorisiert werden (Kontrollflussabhängigkeiten)
		\item Lösung: Bedingt ausgeführte Anweisungen \(\rightarrow\) Umwandlung von Kontrollflussabhängigkeiten in Datenabhängigkeiten
		\item Bedingt ausgeführte Anweisungen
		\begin{itemize}
			\item Vektor-Maskierungssteuerung: Verwendet einen zusätzlichen, boolschen Vektor, um die Ausführung eines Vektorbefehls zu steuern\\\\
				\begin{minipage}{\linewidth}
				\begin{lstlisting}[frame=single,numbers=left,mathescape,language={[mips]Assembler},tabsize=4]
LV 		V1,Ra       ; load vector A into V1
LV 		V2,Rb       ; load vector B
L.D 	F0,#0       ; load FP zero into F0
SNEVS.D V1,F0    	; sets VM(i) to 1 if V1(i)!=F0
SUBV.D 	V1,V1,V2  	; subtract under vector mask
CVM              	; set the vector mask to all 1s
SV 		Ra,V1       ; store the result in A
				\end{lstlisting}
				\end{minipage}
			\item Vektor-Mask-Register: Jede ausgeführte Vektorinstruktion arbeitet nur auf den Vektorelementen, deren Einträge eine \(1\) haben
		\end{itemize}
	\end{itemize}
	\item \textbf{Dünn besetzte Matrizen}
	\begin{itemize}
		\item Elemente eines Vektors werden in einer komprimierten Form im Speicher abgelegt
		\item Indexvektoren zeigen die Elemente an, die nicht \(0\) sind\\\\
			\begin{minipage}{\linewidth}
			\begin{lstlisting}[frame=single,numbers=left,mathescape,language={[mips]Assembler},tabsize=4]
; Berechnet die Summe der duenn besetzten Vektoren A und C.
; Die Indexvektoren K und M zeigen jeweils die Elemente von A und C an,
; die nicht 0 sind
do 	 100 i = 1,n
100  A(K(i)) = A(K(i)) + C(M(i))
			\end{lstlisting}
			\end{minipage}
		\item SCATTER-GATHER-Operationen mit Index-Vektoren: Unterstützen den Transport zwischen gepackter (SCATTER) und normaler (GATHER) Darstellung dünn besetzter Matrizen. Probleme für vektorisierenden Compiler: Konservative Annahmen bzgl. Speicherreferenzen\\\\
			\begin{minipage}{\linewidth}
			\begin{lstlisting}[frame=single,numbers=left,mathescape,language={[mips]Assembler},tabsize=4]
LV 		Vk,Rk 		; load K
LVI 	Va,(Ra+Vk) 	; load A(K(I))
LV 		Vm,Rm 		; load M
LVI 	Vc,(Rc+Vm) 	; load C(M(I))
ADDV.D 	Va,Va,Vc 	; add them
SVI 	(Ra+Vk),Va 	; store A(K(I))
			\end{lstlisting}
			\end{minipage}		
	\end{itemize}
	\item Beispiel: Realisierung von C-Code in Assembler\\\\
		\begin{minipage}{\linewidth}
		\begin{lstlisting}[frame=single,numbers=left,mathescape,language=C,tabsize=4]
unsigned char i;
unsigned char a[64], b[64], c[64];
for (i = 0; i < 64; i++) {
	c[i] = a[i];
	if (a[i] == 0xff) {
		c[i] = b[i];
	}
}
		\end{lstlisting}
		\end{minipage}
		\begin{minipage}{\linewidth}
		\begin{lstlisting}[frame=single,numbers=left,mathescape,language={[mips]Assembler},tabsize=4]
; Initialisierung
MOV 	R1, 64
MTC1	VLR, R1		; Vektorlaenge = 64
LV 		V1, Ra
LV 		V2, Rb
MOV 	R2, 0

; Berechnungen
ADDVS 	V3, V1, R2 	; c[]=a[]
MOV 	R1, 0xff
SEQVS.I	V1, R1		; Vergleich mittels 'equals'
					; Setze Maske gleich 1 bei true, anderenfalls 0
SUBV.I 	V3, V3, V3  ; Setze Werte=0, die in der Maske gesetzt sind
ADDV.I 	V3, V3, V2

; Bereinigen und speichern
CVM					; Vektormaske loeschen, sonst werden nicht alle
					; Werte zurueckgeschrieben
SV 		Rc, V3
		\end{lstlisting}
		\end{minipage}
\end{itemize}

\subsection{SIMD-Verarbeitung in Mikroprozessoren am Beispiel Intel MMX Technologie}
\begin{itemize}
	\item Erweiterung für die IA-32-Prozessorarchitektur, die es erlaubt, größere Datenmengen parallelisiert und somit schneller zu verarbeiten\footnote{\url{https://de.wikipedia.org/wiki/Multi_Media_Extension}}
	\item MMX-Register, die auf FPU-Register abgebildet werden (keine neuen Register eingeführt)
	\item Verarbeitung: (Mehrfach unterteilbares) 64 Bit Datenformat als Basis. Zwei solche Quellen können über eine Funktion verknüpft werden
	\item Saturation Arithemtik: Algorithmen in der grafischen Datenverarbeitung vermeiden Über-/Unterlauf bei Addition und Subtraktion nicht-vorzeichenbehafteter Pixel. Übergang zum größten oder kleinsten darstellbaren Wert. Keine Überprüfung der Werte oder Ausnahmebehandlung
\end{itemize}

