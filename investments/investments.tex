\chapter{Investments}

Zusammenfassung der Vorlesung "`Investments"' von Professor Uhrig-Homburg aus dem Sommersemester 2014.\footnote{https://derivate.fbv.kit.edu/946.php}

\section{Einführung}

\subsection{Kernfrage im Folgenden: Wie sehen optimale Investitionsentscheidungen aus?}
\begin{itemize}
	\item Realinvestitionen: Grundstücke, Gebäude, Maschinen, F\&E-Investitionen
	\item \textbf{Finanzinvestitionen}
	\begin{itemize}
		\item kurzfristige (weitgehend) risikolose Geldanlage
		\item Schuldverschreibungen: Entgelt für Kapitalüberlassung in Form von Zinsen (Anleihen, Bonds)
		\item Aktien: Entgelt für Kapitalüberlassung in Form von Gewinnanteilen
		\item Derivate: Zahlungen, die von orginären Instrumenten abgeleitet werden (Optionen, Futures)
	\end{itemize}
\end{itemize}

\subsection{Portefeuillemanagement}
\begin{enumerate}
	\item \textbf{Festlegen der Investitionspolitik}
	\begin{enumerate}
		\item Spezifische Anlageziele (Portefeuillemanagementziele): Auswahl Ertrags-/Risikocharakteristik
		\item Strategische Asset Allokation (Strategisches PM): Aufteilung der Mittel in Anlageklassen und Wahl des Portefeuillemanagementstils (passiv oder aktiv)
	\end{enumerate}
	\item Durchführen von Finanzanalysen: Finden der besten Anlageinstrumente
	\item \textbf{Portefeuillezusammenstellung und periodische Überwachung}
	\begin{itemize}
		\item Passive Strategie: Investition in Marktindex + risikoloses Papier
		\item \textbf{Aktive Strategie}
		\begin{itemize}
			\item Timing: Investor glaubt besser Markteinschätzung zu haben als der restliche Markt
			\item Picking: Investor akzeptiert Markteinschätzung als Ganzes, glaubt aber einzelne fehlbewertete Instrumente identifiziert zu haben
		\end{itemize}
		$\rightarrow$ Kauf von unterbewerteten Instrumenten (taktisches Portefeuillemanagement)
	\end{itemize}
	\item Performanceanalyse: Analyse des Anlageerfolgs und Zuordnung zu Verlust- und Ertragsquellen
\end{enumerate}


\subsection{Moderne Kapitalmarkttheorien im Überblick}

\subsubsection{Entwicklung der Kapitalmarkttheorien}
\begin{itemize}
	\item Traditionelle Problemstellung: Analyse \textbf{einzelner} Wertpapiere. "`Wie findet man die besten Aktien?"'
	\item Moderne Problemstellung: Analyse von mehreren Wertpapieren
	\begin{itemize}
		\item Sind Kurse prognostizierbar?
		\item Wie soll Portefeuille zusammengestellt werden?
		\item Wo liegen faire Wertpapierpreise?
	\end{itemize}
\end{itemize}

\subsection{Zeit- und Risikodimension von Investitionen}
Abstrakte Formulierung von Investitionen: Tausch einer heutigen Zahlung in spätere Zahlungen.
\begin{itemize}
	\item Zeitdimension: Heute vs. später
	\item Risikodimension: Heute sichere Zahlung vs. zukünftig i.d.R. unsichere Zahlungen
\end{itemize}

\subsubsection{Zentrale Frage: Welche Zahlung heute ist faire Gegenleistung für zukünftige Zahlungen?}
\begin{itemize}
	\item Zeitliche Bewertung
	\item Bewertung des Risikos
\end{itemize}

\subsubsection{Vollkommener Kapitalmarkt}
Ein Kapitalmarkt ist vollkommen, wenn der Preis, zu dem ein Zahlungsstrom zu einem bestimmten Zeitpunkt gehandelt wird, für jeden Marktteilnehmer gleich und gegeben ist.

\subsubsection{Vollständiger Kapitalmarkt}
Ein Kapitalmarkt ist vollständig, wenn jeder beliebiger Zahlungsstrom gehandelt werden kann, ganz gleich, welche Höhe, welche zeitliche Struktur und welche Unsicherheit er aufweist.



\section{Investitionsentscheidungen auf Aktienmärkten}

\subsection{Zentrale Fragestellungen}
\begin{itemize}
	\item Sind Aktienkurse prognostizierbar?
	\item Wie stellt man ein optimales Aktienportefeuille zusammen, wenn man glaubt, den Markt nicht schlagen zu können?
	\item Wie stellen sich faire Aktienkurse ein, wenn sich alle Investoren gemäß der Portefeuilletheorie verhalten?
\end{itemize}


\subsection{Informationseffiziens}
\begin{itemize}
	\item Idee: Marktteilnehmer sammeln und verarbeiten Informationen bzgl. fairen Wertes von Aktien
	\begin{itemize}
		\item Marktpreise der Aktien passen sich fairen Werten an
		\item Marktpreise der Aktien spiegeln verfügbare Informationen wider
	\end{itemize}
	\item Ein Markt ist informationseffizient, wenn die Preise die verfügbaren Informationen vollständig widerspiegeln
\end{itemize}

\subsubsection{Stufen der Informationseffiziens}
Ein Markt ist informationseffizient bzgl. Informationsmenge \(I\), wenn sich Preise durch öffentliche Bekanntgabe der Informationsmenge \(I\) nicht ändern.
\begin{itemize}
	\item Schwache Informationseffiziens: \(I\) enthält alle historischen Kursinformationen
	\item Mittelstrenge Informationseffiziens: \(I\) enthält alle öffentlich verfügbaren Informationen
	\item Strenge Informationseffiziens: \(I\) enthält alle Informationen
\end{itemize}

\subsubsection{Informationsparadoxon}
Anleger smmeln Information, um durch schnelle Reaktionen Vorteile erzielen zu können. Da der Markt sich unendlich schnell an Informationen anpasst, können jedoch keine Vorteile erzielt werden.
\begin{itemize}
	\item Motivation, Informationen zu sammeln lässt nach
	\item Markt spiegelt nicht mehr so schnell alle Informationen wider
\end{itemize}

\subsubsection{Information und Aktienanalyse}
\begin{itemize}
	\item \textbf{Fundamentalanalyse}
	\begin{itemize}
		\item Sammeln und Auswerten von Informationen in Bezug auf Unternehmen und gesamtwirtschaftliches Umfeld
		\item Qualitative Fundamentalanalyse: Unternehmenssituation, Erwartungen über Unternehmensentwicklung
		\item Quantitative Fundamentalanalyse: Ermittlung der exogenen Größen oder Berechnung der exogenen Größen über Bewertungsmodell, die "`inneren"' Wert der Aktie bestimmen
	\end{itemize}
	\item \textbf{Technische Analyse}
	\begin{itemize}
		\item Sammeln und Auswerten von Informationen über Kapitalmarkt, insbesondere historischer Kursdaten
		\item Basiert auf Erfahrungswerten (hist. Marktdaten, Kennzahlen, Charts)
		\item Idee: Kurs-/Umsatzformationen wiederholen sich zyklisch
	\end{itemize}
\end{itemize}


\subsection{Protefeuilletheorie}
Anstatt einzlne Aktien zu untersuchen, nun Analyse von Aktienportefeuilles.

Ökonomische Idee: Durch Erwerb verschiedener Aktien kann Risikominderung erreicht werden (Diversifikation).

\subsubsection{Definition eines Portefeuilles}
Zusammenstellung von Wertpapieren, wobei Umfang der einzelnen Aktien
\begin{itemize}
	\item in Stück \[x = (x_1, x_2, \dots , x_N)\] oder
	\item in Budgetanteilen \[w = (w_1, w_2, \dots , w_N), \sum_{i=1}^{N} w_i = 1\]
\end{itemize}
charakterisiert wird.

Investoren richten sich nach dem \(\mu-\sigma-Prinzip\), d.h. sie wägen erwartete Rendite und Risiko gegeneinander ab.

\subsubsection{Bestimmung effizienter Portefeuilles}

Ausgangssituation: Gegeben \(N\) riskante und ein risikoles Wertpapier.

\subsubsection{Aufteilung in ein riskantes und ein risokoloses Wertpapier}
\begin{itemize}
	\item WP1: \(\mu_1 = r, \sigma_1 = 0\) (risikolos)
	\item WP2: \(\mu_2, \sigma_2 > 0\) (riskant)
	\item Portefeuille \(w = (w_1, w_2), w_1 + w_2 = 1\)
\end{itemize}

$\rightarrow$ \(\tilde{r}_r = w_1 \cdot r + w_2 \cdot \tilde{r}_2\)

\begin{itemize}
	\item Welche Kombinationen sind durch Portefeuillebildung erreichbar?
	\item Welche Portefeuilles sind effizient?
\end{itemize}
Alle erreichbaren \(\mu-\sigma-Kombinationen\) liegen auf einer Geraden.
Portefeuille, das Investor aus der Menge der effizienten als optimales auswählt, hängt von individueller Risikoeinstellung ab (Indifferenzkurven).

\subsubsection{Aufteilung in \(N\) riskante Wertpapiere, \(N = 2\)}
Die erreichbaren \(\mu-\sigma-Kombinationen\) liegen auf einer Hyperbel.
Alle erreichbaren Portefeuilles, die nicht dominiert werden (oberhalb des \textit{globalen varianzminimalem Portfeuille}) sind effizient (oberhalb des Extrempunkts).

\subsubsection{Zusammenfassung}
\begin{itemize}
	\item Die \(\mu-\sigma-Kombinationen\) liegen auf einer Hyperbel, falls \(|\rho_{12}| < 1\), anderenfalls liegen sie auf einer Geraden
	\item Die effizienten Portefeuilles liegen oberhalb des GVMP
	\item Ohne Leerverkaufsbeschränkung ist das Risiko des GVMP immer kleiner als die Einzelrisiken $\rightarrow$ grundsätzliche Risikoreduktion durch Diversifikation
	\item Das optimale Portefeuille hängt von der individuellen Risikoeinstellung des Investors ab
\end{itemize}

\subsubsection{Aufteilung in \(N\) riskante Wertpapiere, \(N > 2\)}
\begin{itemize}
	\item Die Menge der erreichbaren \(\mu-\sigma-Kombinationen\) bilden eine Fläche
	\item Der Rand ist durch die diejenigen \(\mu-\sigma-Kombinationen\) charakterisiert, die bei gegebenem \(\mu\) das kleinstmögliche \(\sigma\) aufweisen
	\item Der Teil des Rands oberhalb des GVMP bilden den effizienten Rand
\end{itemize}


\subsection{Index-Modelle}

\subsubsection{Markowitz-Modell}
Investor benötigt je nach Anlagespektrum viele Schätzwerte $\rightarrow$ erheblicher Schätzaufwand.

Ziel im Folgenden: Durch Vorgabe einer Struktur hinsichtlich Risiken den Schätzaufwand reduzieren.

\subsubsection{Single Index Modell}
Aktienrendite wird an einen Faktor (Marktindex) gekoppelt.
\[\tilde{r}_i = \alpha_i + \beta_i \cdot \tilde{r}_I + \tilde{\epsilon}_i\]

\subsubsection{Strukturelle Annahmen}
\begin{enumerate}
	\item Aktienindividuelle Störterme sind nicht systematisch verzerrt
	\item Aktienindividuelle Störterme sind unkorreliert zur Marktrendite
	\item Aktienindividuelle Störterme sind paarweise unkorreliert
\end{enumerate}

\begin{itemize}
	\item \(\beta_i^2 > 1\) $\rightarrow$ aggresive Aktie
	\item \(\beta_i^2 < 1\) $\rightarrow$ defensive Aktie
\end{itemize}

\subsubsection{Implikationen}
\begin{itemize}
	\item Indexentwicklung einzige Quelle für Korrelationen zwischen Aktienrenditen
	\item Risiko einzelner Aktien kann in \textit{nicht diversifizierbares} und \textit{Residualrisiko} eingeteilt werden
	\item Anzahl der Parameter deutlich geringer als bei vollem Markowitz-Modell
	\item Portefeuille-Beta entspricht der gewichteten Summer der Einzel-Betas
	\item Restrisiko entspricht der gewichteten Summer der Einzelrestrisiken
	\item Restrisiko kann im Gegensatz zum Marktrisiko reduziert werden
\end{itemize}


\subsection{Capital Asset Pricing Modell}

\subsubsection{Modellannahmen}
\begin{enumerate}
	\item \textbf{Modellannahmen}
	\begin{enumerate}
		\item \textbf{Investoren}
		\begin{itemize}
			\item Wählen Portefeuille auf Basis von Erwartungswert und Varianz der Rendite
			\item Besitzen homogene Erwartungen bzgl. relevanter Entscheidungsparameter
			\item Besitzen den selben Anlagehorizont (im Folgenden 1 Jahr)
		\end{itemize}
		\item \textbf{Kapitalmärkte}
		\begin{itemize}
			\item Sind friktionslos bei vollkommener Konkurrenz
			\item Risikoloses Paper exisiert
			\item Bestehen aus handelbaren Wertpapieren bei festem Angebot
		\end{itemize}
	\end{enumerate}
	\item \textbf{Individualkalkül der Investoren}
	\begin{itemize}
		\item Jeder Investor optimiert gemäß Portefeuilletheorie
		\item Tangentialportefeuilles der einzelnen Anleger simmen in relativer Zusammensetzung der riskanten Instrumente überein
	\end{itemize}
\end{enumerate}

\subsubsection{Marktportefeuille}
Portefeuille bestehend aus allen umlaufenden Aktien des betrachteten Marktes in den relativen Anteilen ihres Wertes.

\subsubsection{Ergebnis im Gleichgewicht}
\begin{itemize}
	\item Jeder Investor legt sein riskantes Vermögen in der gleichen Zusammensetzung wie das Marktportefeuille an
	\item Die risikolose Komponente wird durch den Grad der individuellen Risikoaversion bestimmt
\end{itemize}

\subsubsection{Kapitalmarktlinie}
Alle effizienten Portefeuilles liegen auf einer Geraden (Capital Market Line).
\begin{itemize}
	\item Marktpreis des Risikos gibt an, wieviel zusätzliche Rendite im Gleichgewicht für zusätzliches Risiko zu erwarten ist
	\item Das einzige effiziente Aktienportefeuille ist das Marktportefeuille
	\item Auf Basis von CAPM ist passive Handelsstrategie (Nachbilden des Marktportefeuilles) sinnvoll
\end{itemize}

\subsubsection{Zerlegung des Gesamtrisikos \(\sigma_i^2\) der einzelnen Aktie}
\begin{itemize}
	\item Risikobeitrag zum Gesamtrisiko \(cov(\tilde{r}_i, \tilde{r}_M)\): Nicht diversifizierbares systematisches Risiko
	\item Unternehmensbezogenes Restrisiko \(\sigma_{\epsilon_i}^2\): Diversifizierbares unsystematisches Risiko. Übernahme dieses Risikos wird vom Markt nicht durch höhere erwartete Rendite vergütet
	\item Zerlegung: \(\sigma_i^2 = \beta_i^2 \cdot \sigma_M^2 + \sigma_{\epsilon_i}^2\)
\end{itemize}
$\rightarrow$ Anstatt \(\sigma_i\) ist \(cov(\tilde{r}_i, \tilde{r}_M)\) als Risikomaß relevant.

\subsubsection{Zentrale Aussage des CAPM}
Für alle Wertpapiere gilt eine lineare Beziehung zwischen erwarteter Rendite und Risiko.

Der risikolose Zinssatz entspricht der erwarteten Rendite des Zero-Beta-Portefeuilles, da im CAPM nur systematisches Risiko vergütet wird.


\subsection{Arbitrage Pricing Theory}
\begin{itemize}
	\item Empirische Überprüfbarkeit des CAPM in Frage gestellt
	\item Alternative Bewertungstheorie des CAPM, die Marktportefeuille nicht benötigt
	\item \textbf{Prinzip}
	\begin{itemize}
		\item Anstatt \(\mu - \sigma - Optimierer\) zu unterstellen, renditegenerierter Prozess: \(\tilde{r}_i = \alpha_i + \beta_{i,1} \cdot \tilde{I}_1 + \alpha_2 + \beta_{i,2} \cdot \tilde{I}_2 + \dots + \alpha_i + \beta_{i,k} \cdot \tilde{I}_k + \tilde{\epsilon}_i\)
		\item Marktpreise der Faktorrisiken: \(\lambda_1, \dots, \lambda_k\)
		\item Falls risikoloses Papier existiert: \(\lambda_0 = r\)
	\end{itemize}
\end{itemize}


\subsection{Performancemessung}
Wie kann die Performance
\begin{itemize}
	\item eines Portefeuilles (Handelsstrategie)\\
	bzw
	\item einer Fondinvestition
\end{itemize}
gemessen und beurteilt werden?

\subsubsection{Anwendung}
\begin{itemize}
	\item leistungsgerechte Vergütung des Fondsmanager
	\item Grundlage für zukünftige Investitionsentscheidungen
\end{itemize}

\subsubsection{Performancemessung auf Basis der Rendite}
\begin{itemize}
	\item Wenig aussagekräftig, da sie sowohl auf Zufall, Übernahme von (hohen) Risiken oder besonderen Fähigkeiten des Fondsmanager beruhen kann
	\item Besser: Risikoadjustierte Rendite oder Analyse auf Basis langer Zeitreihen
\end{itemize}

\subsubsection{Sharpe-Ratio}
Idee: Investoren präferieren das Wertpapier, bei dem sich durch Kombination mit dem risikolosen die beste Rendite/Risikokombination erreichen lässt.

\subsubsection{Jensen's Alpha}
Idee: Vertikaler Abstand der erzielten Durchschnittsrendite von der Gleichgewichtsrendite gemäß eingegangenem Risiko auf der Wertpapiermarktlinie.

Berücksichtigt allerding nicht, dass gewünschtes Risiko durch Mischung aus riskant und risikolos erzeugt werden kann.

\subsubsection{Treynor-Maß}
Idee: Mittlere Überrendite pro Einheit systematisches Risiko.



\section{Investitionsentscheidungen auf Rentenmärkten}

\subsection{Renditekennziffern}
Analyse festverzinslicher Wertpapiere erfolgt i.A. auf Basis verschiedener Kennziffern, die im Grundsatz Ertrag und Risiko des Wertpapiers quantifizieren.

\subsubsection{Nullkuponanleihe}
Eine Nullkuponanleihe  ist eine Sonderform des verzinslichen Wertpapiers. Dabei gibt es keinen Kupon (d. h. keine laufende Zinszahlung) und nur eine Auszahlung am Ende der Laufzeit der Anleihe. Der Gewinn für den Anleger besteht damit nur in der Differenz zwischen dem Erwerbskurs und dem Rückzahlungspreis bzw. Verkaufskurs.\footnote{\url{http://de.wikipedia.org/wiki/Nullkuponanleihe}}

\subsubsection{Yield to maturity}
Die Effektivverzinsung (Yield to Maturity, YtM) errechnet sich aus der Diskontierung der zukünftigen Cash Flows (Kupon und Nominalbetrag) mit einem einheitlichen Diskontierungsfaktor. Ergebnis der Diskontierung ist der heutige Kurs.\footnote{\url{http://de.wikipedia.org/wiki/Yield_To_Maturity}}

\subsubsection{Plain Vanilla Zinsswaps}
\begin{itemize}
	\item Vereinbarung zweier Parteien, Zinszahlungen zu tauschen
	\item Eine Partei zahlt Zinsen bezogen auf festen Zinssatz (Swapsatz) und festgelegten Nominalbetrag (Swapkäufer, Fix Payer)
	\item Zweite Partei zahlt Zinsen bezogen auf variablen Zinssatz und denselben Nominalbetrag (Swapverkäufer, Fix Receiver)
\end{itemize}

\begin{tabularx}{\columnwidth}{l|X|X|X|c|l}
						& \(t=0\) 	& \(t=1\)		& \(t=1\)			& \(\dots\) 	& \(t=T\) \\
	\hline
	Fix Payer zahlt 	& \(0\)		& \(s(0,T)\)	& \(s(0,T)\)		& \(\dots\)		& \(s(0,T)\) \\
	Fix Receiver zahlt 	& \(0\)		& \(L(0)\)		& \(\tilde{L}(1)\)	& \(\dots\)		& \(\tilde{L}(T-1)\)
\end{tabularx}

Alle Änderungen innerhalb einer Periode wirken sich erst auf die folgende Periode aus.

\subsubsection{Zusammenfassung}
\begin{itemize}
	\item Interne Renditen von Kuponanleihen
	\begin{itemize}
		\item Hängen von der Kuponhöhe ab
		\item Eine Anlageentscheidung aufgrund der internen Rendite ist wenig sinnvoll
		\item Vielmehr ist zu prüfen, ob die Kuponanleihe relativ zur Zinsstrukturkurve fair bewertet ist
	\end{itemize}
\end{itemize}


\subsection{Duration}
Die Duration ist eine Sensitivitätskennzahl, die die durchschnittliche Kapitalbindungsdauer einer Geldanlage in einem festverzinslichen Wertpapier bezeichnet. Genauer genommen und allgemein formuliert ist die Duration der gewichtete Mittelwert der Zeitpunkte, zu denen der Anleger Zahlungen aus einem Wertpapier erhält.\footnote{\url{http://de.wikipedia.org/wiki/Duration}}

Prinzipiell basiert das Risikomaß Duration auf der Grundidee, dass das Zinsrisiko direkt mit der Restlaufzeit zusammenhängt. So ist das Risiko umso höher, je länger eine Anleihe läuft.\footnote{\url{http://finance.wiwi.tu-dresden.de/Wiki-fi/index.php/Duration}}
\begin{itemize}
	\item Macaulay-Duration: Zeitgewichtete diskontierte Zahlungen pro Preiseinheit, relatives Maß für die Wertänderung
	\item EUR-Duration: Änderung eines Wertpapierpreises bei einer Zinsänderung von einem Prozentpunkt, Maß für die absolute Preisänderung
\end{itemize}



\section{Appendix A: Begrifflichkeiten}

\subsubsection{Festverzinslicher Titel}
Schuldrechtlicher Anspruch auf
\begin{itemize}
	\item Zahlung eines zeitabhängigen Entgelts (Zinsen)
	\item Rückzahlung des überlassenen Kapitals
	\item Beispiele: Darlehen, Anleihe
\end{itemize}

\subsubsection{Anleihen}
\begin{itemize}
	\item Werden in homogene Teile zerstückelt und verbrieft (Homogenität ermöglicht Börsenhandel)
	\item Werden typischerweise am Sekundärmarkt gehandelt
	\item Besitzen unterschiedliche Ausstattungsmerkmale (Art der Verzinsung/Rückzahlung, Rückzahlungskurs, Laufzeit)
\end{itemize}

\subsubsection{Termingeschäfte}
\begin{itemize}
	\item Terminpreis einer Nullkuponanleihe \(F(0, S, T)\): Zum Zeitpunkt \(0\) vereinbarter Preis über den unbedingten Kauf einer in \(T\) fälligen Nullkuponanleihe im Zeitpunkt \(S\)
\end{itemize}

\subsubsection{Zinsen}
\begin{itemize}
	\item Kassazins (spot price): Zinssatz für eine Kapitalanlage, deren Laufzeit sofort beginnt
	\item Terminzins (forward rate): Zinssatz, der für einen zukünftigen Zeitpunkt gilt
\end{itemize}



\section{Appendix B: Formelsammlung}

\subsection{Portefeuilletheorie}

\subsubsection{Rendite einer Aktie}
\[\tilde{r} = \frac{\tilde{S} (1) + D - S(0)}{S(0)},~\mu_i = \frac{E[S_i(1)]}{S_i(0)} - 1,~\sigma_i = \frac{Std[S_i(1)]}{S_i(0)}\]

\subsubsection{Portefeuillerendite}
Gewichtete Summe der Einzelrenditen. Gewichtung am Wertanteil \(w_i\) am Gesamtwert des Portefeuilles.

\subsubsection{Erwartete Portefeuillerendite}
\[\mu_w = \sum_{i=1}^{N} w_i \cdot \mu_i\]

\subsubsection{Varianz der Portefeuillerendite}
\[\sigma_w^2 = \sum_{i=1}^{N}\sum_{j=1}^{N} w_i \cdot w_j \cdot \rho_{ij} \cdot \sigma_i \cdot \sigma_j\]

\subsubsection{Erreichbare \(\mu-\sigma-Kombinationen\)}
\[\mu_w = r + \frac{\mu_2 - r}{\sigma_2} \cdot \sigma_w \]

\subsubsection{Berechnung der Geradengleichung der effizienten Linie \(\mu_w(\sigma_w)\)}
\[\vec{\mu}_x = \vec{\mu} - \vec{r}\]
\[V = \begin{pmatrix} \sigma_1^2 & 0 \\ 0 & \sigma_2^2 \\ \end{pmatrix}\]
\[\vec{w}_M = V^{-1} \cdot \vec{\mu}_x\]
\[\mu_M = w_M^T \cdot \vec{\mu}_x - r\]
\[\sigma_M = \sqrt{\vec{w}_M^T \cdot V \cdot \vec{w}_M}\]

\subsubsection{Aufteilung in \(N\) riskante Wertpapiere, \(N = 2\), Bestimmung des GVMP}
\[\tilde{r}_w = w_1 \cdot \tilde{r}_1 + w_2 + \tilde{r}_2\]
\[\mu_w = w_1 \cdot \mu_1 + w_2 \cdot \mu_2\]
\[\sigma_w = \sqrt{w_1^2 \cdot \sigma_1^2 + 2 \cdot \sigma_1 \cdot \sigma_2 \cdot \rho_{12} \cdot w_1 \cdot w_2 + w_2^2 \cdot \sigma_2^2}\]
\[w_1 = \frac{\sigma_2^2-\sigma_1 \cdot \sigma_2 \cdot \rho_{12}}{\sigma_1^2-2 \cdot \sigma_1 \cdot \sigma_2 \cdot \rho_{12} + \sigma_2^2}\]
\[w_2 = \frac{\sigma_1^2 - \sigma_1 \cdot \sigma_2 \cdot \rho_{12}}{\sigma_1^2 - 2 \cdot \sigma_1 \cdot \sigma_2 \cdot \rho_{12} + \sigma_2^2}\]
\[\sigma_w^2 = \frac{\sigma_1^2 \cdot \sigma_2^2 \cdot (1 - \rho_{12}^2)}{\sigma_1^2 - 2 \cdot \sigma_1 \cdot \sigma2 \cdot \rho_{12} + \sigma_2^2}\]
\[\sigma_w = \sqrt{Var[r_w]} = \frac{1}{S(0)} \cdot \sqrt{Var[S(1)]} \]

\subsubsection{Single Index Modell}
\[\mu_i = \alpha_i + \beta_i \cdot \mu_I\]
\[\sigma_i^2 = \beta_i^2 \cdot \sigma_I^2 + \tilde{\sigma}_{\epsilon_i}\]
\[co(v\tilde{r}_i, \tilde{r_j}) = \beta_i \cdot \beta_j \cdot \sigma_I^2\]


\subsection{Capital Asset Pricing Modell}
\[\mu_w = r + \frac{\mu_M - r}{\sigma_M} \cdot \sigma_w = r + (\mu_M - r) \cdot \beta_j\]
\[\beta_w = \frac{\sigma_w}{\sigma_M} = \frac{cov(r_w, r_M)}{\sigma_M^2} = \frac{\sigma_w \cdot \sigma_M \cdot \rho_{wM}}{\sigma_M^2} \]

Aus negativer Korrelation folgt, dass \(\beta_w\) negativ ist, woraus wiederum folgt, dass \(\mu_w < r\).

\subsubsection{Systematisches Risiko}
CAPM ohne systematisches Risiko: \(\beta_{Pf} \stackrel{!}{=} 0\)


\subsection{Arbitrage Pricing Model}

\subsubsection{Rendite}
\[\tilde{r}_i = \alpha_i + \beta_{i,1} \cdot \tilde{I}_1 + \alpha_2 + \beta_{i,2} \cdot \tilde{I}_2 + \dots + \alpha_i + \beta_{i,k} \cdot \tilde{I}_k + \tilde{\epsilon}_i\]


\subsection{Performancemessung}

\subsubsection{Sharp-Ratio}
\[Sharp-Ratio = \frac{Mittlere~Ueberrendite}{Gesamtrisiko} = \frac{\overline{\mu} - r}{\sigma}\]

\subsubsection{Jensen's Alpha}
\[\alpha = (\overline{\mu} - r) - (\mu_M - r) \cdot \beta\]

\subsubsection{Treynor-Maß}
\[Treynor-Mass = \frac{Mittlere~Ueberrendite}{systematisches~Risiko} = \frac{\overline{\mu} - r}{\beta}\]


\subsection{Anleihen}

\subsubsection{Begriffe}
\begin{itemize}
	\item Diskontfaktor: \(b(0,T)\)
	\item Effektivzins oder Kassazins: \(y(0,T)\)
	\item Arbitagefrei: Die Diskontstrukturkurve \(b(0,T)\) ist monoton fallend
\end{itemize}

\subsubsection{Nullkuponanleihe}
\[Barwert = \frac{Nennwert}{(1+i)^t} \iff i = \sqrt[t]{\frac{Nennwert}{Barwert}} -1\]

\subsubsection{Rendite einer Kuponanleihe}
\[P(0) = \sum_{t=1}^{T} \frac{C}{(1+y(0,t))^t} + \frac{RZK}{(1+y(0,T))^T}\]

\subsubsection{Berechnung Diskontfaktor}
\[b(0,T) = \frac{1}{(1+y(0,T))^T}\]

\subsubsection{Berechnung Swapsatz}
Der Swapsatz entspricht dem Kupon einer zu pari notierenden Kuponanleihe.
\[1 = \sum_{t=1}^{T} \frac{s(0,T)}{(1+y(0,t))^t} + \frac{1}{(1+y(0,T))^T}\]

\subsubsection{Terminzinssatz}
\[(1+f(0,S,T))^{T-S} = \frac{(1+y(0,T))^T}{(1+y(0,S))^S}\]


\subsection{Duration}
\[P(0) = \sum_{t=1}^{T} \frac{c \cdot 100}{(1+y(0,t))^t} + \frac{RZK}{(1+y(0,T))^T}\]

\subsubsection{Macaulay Duration}
\[D^{Mac} = \frac{\sum_{t=1}^{T} t \cdot \frac{c \cdot 100}{(1+z)^t} + \frac{T \cdot RZK}{(1+z)^T}}{P(0)} \]

\subsubsection{EUR Duration}
\[D^{EUR} = D^{Mac} \cdot \frac{1}{1+z} \cdot P(0) \]

