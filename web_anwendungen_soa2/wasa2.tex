\chapter{Web-Anwendungen und Serviceorientierte Architekturen (II)}

Zusammenfassung der Vorlesung "`Web-Anwendungen und Serviceorientierte Architekturen (II)"' aus dem Sommersemester 2017.\footnote{\url{https://cm.tm.kit.edu/study_wasa2p.php}}

\section{C\&M-Software-Entwicklung}
\begin{itemize}
	\item Ziele: Stetige Weiterentwicklung der praktizierten Software-Entwicklung; nachvollziehbar, modellgestütz, agil
	\item Artefakte, die während den Entwicklungsphasen entstehen: Sammlungen, Domänenmodelle, API-Spezifikationen, Qualitätsberichte, Implementierungen
	\item Jedes Artefakt wird auf verschiedenen Abstraktionsebenen dokumentiert (Entwicklungsdokument, Artefakt-Dokument, Entwicklungswerkzeug)
	\item \textbf{Schichtenmodell und eingesetzte Technologien}
	\begin{description}
		\item[Frontend:] Angular, Bootstrap
		\begin{description}
			\item[] Präsentation
			\item[] Präsentationslogik
		\end{description}
		\item[Backend-For-Frontend:] Spring
		\begin{description}
			\item[] Anwendungslogik
		\end{description}
		\item[Backend:] Spring
		\begin{description}
			\item[] Domänenlogik
			\item[] Datenhaltung
		\end{description}
	\end{description}
	\item \textbf{Entwicklungsumgebung}
	\begin{itemize}
		\item C\&M-Team-Server: Dokumentenablage, Backlog, Profile, Termin-Kalender
		\item C\&M-Team-Foundation-Server: Backlogs mit UserStories, Code-Verwaltung (Git)
		\item IDE: Eclipse mit Git-Client
		\item Werkzeuge: MS Office, UI-Werkzeuge, Enterprise Architect
	\end{itemize}
	\item Rollen und Aufgaben in der Software-Entwicklung: Betreuer, BA/MA-SeniorStudent, JuniorStudent
	\item Humane Service Registry: Zentrale Übersicht aller Services
	\item Leitfaden: C\&M-konformes Dokument zur praktischen Einführung in eine Technologie/Service/Konzept; beinhalten Technologiegrundlagen und Umsetzungsbeispiele
	\item Eigenentwicklungsdokument (EED): Am Lehrstuhl entwickelte Software
\end{itemize}



\section{Domain Modeling}

\subsection{Einführung}
\begin{itemize}
	\item Motivation: Die Komplexität von Domänen-Wissen erschwert die Software-Entwicklung
	\item \textbf{Änderungen am konventionellen Entwicklungprozess}
	\begin{description}
		\item[Design:] Domain-Modelling findet vor dem eigentlichen Design-Vorgang statt (nach der Analyse-Phase)
		\item[Implementierung:] Bildet die Geschäftslogik im Backend zwischen Infrastruktur-Ebene und Backend-Service-API
	\end{description}
	\item \textbf{"`Knowledge Crunching"'}
	\begin{itemize}
		\item Domänenexperten und Entwickler erarbeiten gemeinsam ein Modell
		\item Best Practices
		\begin{itemize}
			\item Entwickeln von Prototypen zur Verständnisvalidierung
			\item Entwickeln einer domänen-spezifischen "`Sprache"'
			\item Verhaltensmodellierung, nicht nur Datenmodellierung
			\item Dynamischer Prozess, der iteratives Diskutieren und Testen beinhaltet
		\end{itemize}
	\end{itemize}
	\item Ubiquitous Language: Gemeinsame Sprache, die technische und betriebswirtschaftliche Ausdrücke vereint (Schnittmenge); wird von allen Teammitgliedern verwendet; oft mehrere verschiedene Diagramme zum Beschreiben eines Sachverhalts notwendig
	\item \textbf{Hexagonale Architektur}
	\begin{itemize}
		\item Ziel: Isoliertes Entwickeln und Testen von Anwendungen, ohne Geräte/Datenbanken/Benutzern \(\rightarrow\) Trennung zwischen Geschäftslogik und externen Entitäten
		\item Anwendung soll unabhängig von ihrer Laufzeitumgebung testbar sein
		\item Realisiert mittels \textit{Ports} und \textit{Adaptern}: \textit{Adapter} konvertieren \textit{Events}, die an \textit{Ports} eintreffen in Methoden
		\item Hexagonale Darstellung symbolisiert die verschiedenen Schnittstellen
	\end{itemize}
\end{itemize}


\subsection{Domain-Driven Design (DDD)}
\begin{itemize}
	\item Wiederverwendung existierender Konzepte (Objektorientierung, Responsibility-Driven-Design, etc.)
	\item \textbf{Pattern: Layered Architecture}
	\begin{itemize}
		\item Ziel: Separierung der Geschäftslogik von anderen Komponenten \(\rightarrow\) einfache Wartbarkeit der Geschäftslogik
		\item Kommunikation bei Schichtenarchitekturen: Direkte Aufruf der darunter liegenden Schicht; indirekte Kommunikation mit der Schicht darüber (Callbacks, Observer-Pattern, etc.)
		\item Die Schichten
		\begin{description}
			\item[User Interface:] Benutzeransicht mit Kommondoeingabe
			\item[Application Layer:] Aufgabenkoordinierung
			\item[Domain Layer:] Geschäftslogik mit Informationsspeicher
			\item[Infrastructure:] Technische Infrastruktur
		\end{description}
		\item Anforderungen an die Schichtenarchitektur
		\begin{itemize}
			\item Lose Koppling der Schichten: Abhängigkeiten nur nach unten
			\item Trennung der Domänenschicht von den anderen
			\item Trennung von Infrastrukturschicht und Domänenschicht
			\item Gezielte Auswahl von Frameworks
		\end{itemize}
	\end{itemize}
	\item \textbf{Entity, Value, Object, Service und Module}
	\begin{itemize}
		\item Objekttypen
		\begin{description}
			\item[Entity:] Attributobjekt, das zur Identifikation eines Objekts benutzt werden kann (\texttt{objectId}, etc.)
			\item[Value Object:] Attributobjekt, das weitergehende Informationen speichert, die nicht zur Identifikation benutzt werden können (\texttt{address}, etc.)
		\end{description}
		\item Service: Zustandslose Operation, die von einem Interface bereitgestellt wird
		\item Module: Fasst Teile der Domäne lose gekoppelt zusammen
	\end{itemize}
	\item \textbf{Aggregate}
	\begin{itemize}
		\item Fasst Objekte zu einem komplexen Objekt zusammen
		\item Zugriff auf das Aggregat lediglich über das Root-Element (vgl. Interface)
	\end{itemize}
\end{itemize}


\subsection{Anwendung in der Software-Entwicklung}
\begin{itemize}
	\item \textbf{Domain Modeling Guideline}
	\begin{itemize}
		\item Ziel: Organisationseinheitliche Anwendung von DDD-Pattern
		\item Beinhaltet: Repository-Struktur der Artefakte inklusive Werkzeuge zum Ändern der Artefakte; strukturelle Beschreibung der Modellierungsschritte
	\end{itemize}
	\item \textbf{Strukturierung einer Domäne}
	\begin{itemize}
		\item Unterteilung in Sub-Domänen (Pakete). Beispiele: \texttt{HazardWarningGroup} beinhaltet die Sub-Domänen \texttt{Hazard} und \texttt{Vehicle}
		\item Beziehungen zwischen Paketen werden durch geteilte Objekte (Entities oder Value Objects) ausgedrückt (beispielsweise Typ \texttt{Event})
	\end{itemize}
	\item \textbf{Domain Objekten}
	\begin{itemize}
		\item Verwendung von \textit{Domänen Objekten}, um die Domäne zu modellieren
		\item Beziehungstypen: Association, Aggregation (Ziel kann alleine existieren), Composition (Ziel kann nicht alleine existieren), Generalization (Vererbung)
	\end{itemize}
	\item \textbf{Domain Views}
	\begin{itemize}
		\item Ziel: Modellierung einer Domain aus verschiedenen Perspektiven
		\item Statische Aspekte
		\begin{description}
			\item[Relation View:] Statische Konzepte der Domänen-Objekte
			\item[Domain Object View:] Detailliere Informationen zu Domänen-Objekten
			\item[Interaction View:] Statische Beschreibung der Beziehungen zwischen Domänen-Objekten
		\end{description}
		\item Dynamische Aspekte
		\begin{description}
			\item[Process View:] Zeitliche Aspekte von Interaktionen zwischen Domänen-Objekten. Beispielsweise ein Sequenzdiagramm, das den Benutzeranlege-Prozess beschreibt
		\end{description}
	\end{itemize}
\end{itemize}



\section{Microservices und Web-\texttt{APIs}}

\subsection{Einführung}
\begin{itemize}
	\item Motivation
	\begin{itemize}
		\item Monolithische Software-Systeme werden immer größer und komplexer (siehe "`Software-Krise"' von 1968)
		\item \textit{System-Zentrierung}: Software-Entwicklung, da hier immer ein Software-System realisiert wird
		\item \textit{Prozess-Zentrierung}: SOA-Entwicklung, da Geschäftsprozesse unterstützt werden sollen
	\end{itemize}
	\item Feingranulare, kooperierende Services, die jeweils individuelle Teilaufgaben übernehmen und individuell entwickelt und deployt werden
	\item \textbf{Ursprünge}
	\begin{itemize}
		\item Domain-Driven Design
		\item Hexagonale Architektur
		\item Continuous Delivery (CD) und Build Pipelines
		\begin{itemize}
			\item Jeder Repository-Checkin wird als Release Candidate behandelt, automatisch gebaut und automatisiert getestet
			\item Visualisierung des Zustands
			\item Eine Pipeline pro Microservice
		\end{itemize}
		\item Virtualisierung
		\begin{itemize}
			\item System-Virtualisierung: Hyperviser verwaltet die Gastressourcen. Pro VM ein Betriebssystem mit eigenem Kernel
			\item Container-Virtualisierung: Prozessvirtualisierung mit separatem Prozess pro Container. Ohne Hypervisor; erlaubt schnelleres Provisioning; Container teilen sich den Kernel
		\end{itemize}
	\end{itemize}
	\item \textbf{Beziehung zu \texttt{SOA}}
	\begin{itemize}
		\item Microservices bilden einen bestimmten Ansatz um \texttt{SOA} umzusetzen
		\item Verwendung von Shared Libraries empfehlenswert, erhöhen allerdings auch die Kopplung
	\end{itemize}
\end{itemize}


\subsection{Grundlagen}
\begin{itemize}
	\item \textbf{Eigenschaften/Ziele}
	\begin{itemize}
		\item "`Single-Responsible-Principle"': Jeder Microservice implementiert einen bestimmten Prozess. Idealerweise ein Microservice pro Team, Entwicklungsaufwand sollte zwei Wochen nicht übersteigen
		\item Selbstständigkeit: Ein Microservice pro Gerät \(\rightarrow\) Änderungen am Microservice bleiben auf eine Maschine beschränkt
		\item Kommunikation über \texttt{APIs}
		\item Lose Kopplung: Änderungen an einem Service dürfen keine Auswirkungen auf andere Services haben \(\rightarrow\) reduziert den Verwaltungsaufwand
		\item Hohe Kohäsion: Für jede Aufgabe ist genau ein Service zuständig \(\rightarrow\) Änderungen betreffen lediglich einen Service (siehe "`Single-Responsible-Principle"')
	\end{itemize}
	\item \textbf{Backends for Frontends (BFFs)}
	\begin{itemize}
		\item Serverseitige Backends für UIs (ein Backend pro UI), die unabhängig von einander entwickelt werden können
		\item Weiterentwicklung von API-Gateways
		\item Idealerweise ein Backend pro Team
	\end{itemize}
	\item \textbf{Kommunikationsmöglichkeiten}
	\begin{itemize}
		\item Request-Response-basiert
		\begin{itemize}
			\item Remote Procedure Call (RPC): Stub-basierter Zugriff auf eine entfernte Ressource. Bsp.: \texttt{SOAP}
			\item REpresentational State Transfer (REST): Ressourcen-orientiert und vom internen Speicher entkoppelt; basiert auf \texttt{HTTP}. RESTful \texttt{APIs} müssen die fünf REST-Prinzipien erfüllen
			\begin{enumerate}
				\item Adressierbarkeit von Ressourcen über eindeutige \texttt{URLs}
				\item Triviale Verwaltungsmethoden von Ressourcen: \texttt{CRUD}
				\item Zustandslose Kommunikation
				\item Verschiedene Repräsentationen der Ressourcen, beispielsweise \texttt{HTML}, \texttt{JSON} oder \texttt{XML}; definierbar via \texttt{Content-Type} im \texttt{HTTP}-Header
				\item Navigation ausschließlich über \texttt{URLs}, die vom Server bereitgestellt werden (Hypermedia) \(\rightarrow\) erhöht die lose Kopplung, da hartkodierte Pfade vermieden werden
			\end{enumerate}
		\end{itemize}
		\item Event-basiert (Producer-Consumer-Schema): \textit{Message-Brokers} verteilten Event-Nachrichten. Beispiele: \texttt{RabbitMQ} oder \texttt{Apache Kafka}
	\end{itemize}
	\item \textbf{Klassifizierung von \texttt{APIs} nach Richardson}
	\begin{itemize}
		\item Vier verschiedene, hierarchisch angeordnete Typen, die jeweils auf den untergeordneten basieren
		\begin{description}
			\item{Level 3:} Hypermedia (Hypermedia-orientiert)
			\item{Level 2:} Protocol Properties (Ressourcen-orientiert; eindividuelle \texttt{URLs}, Nutzung der \texttt{HTTP}-Methoden)
			\item{Level 1:} Resources (Ressourcen-orientiert; individuelle \texttt{URL} pro Ressource, aller per \texttt{POST})
			\item{Level 0:} Plain Old XML; einzelne \texttt{URL}, aller per \texttt{POST}
		\end{description}
	\end{itemize}
\end{itemize}


\subsection{Design von Web-APIs}
\begin{itemize}
	\item Qualitätskriterien: Einfach zu verstehen; Technologie-unabhängig; konsistent; erweiterbar; weiterentwickelbar
	\item \texttt{API} sollte Startpunkt der Entwicklung sein ("`\texttt{API} first"'); Design-Guidelines aufstellen und umsetzen
	\item \textbf{Entwicklungsstrategie: \texttt{API} First}
	\begin{enumerate}
		\item Anforderungen an den Service aufstellen (ableiten aus Analysephase)
		\item Iterativ: Entwickeln/Verbessern der \texttt{API}; Reviewen der \texttt{API}
		\item Implementierung von Service und ServiceConsumer
	\end{enumerate}
	\item \textbf{\texttt{API} Guideline-Sammlung}
	\begin{itemize}
		\item Ressourcen-orientiert mit einheitlichem Look-and-Feel
		\item Einheitliche Namenssyntax, beispielsweise bei \texttt{HTTP}-Headern oder \texttt{JavaScript}-Objekten
		\item Selbsterklärende Beschreibungen bei URLs zur Ressourcen-Identifikation
		\item Sinnvolle Objektfilter. Bsp.: \texttt{/sales-orders/?filter=(creation\_date=20151106)}
		\item Korrekte Verwendung der \texttt{HTTP}-Methoden
	\end{itemize}
	\item \textbf{OpenAPI}
	\begin{itemize}
		\item Spezifikation zur Definition von RESTful \texttt{APIs}; unterstützt von einigen IT-Konzernen; aus Swagger abgeleitet
		\item \texttt{JSON}- oder \texttt{YAML}-Bschreibung
		\begin{description}
			\item[info:] Metainformationen
			\item[paths:] Endpunkte und Methoden
			\item[definitions:] Zur Wiederverwendung von Objekten
		\end{description}
	\end{itemize}
	\item \textbf{Swagger}
	\begin{itemize}
		\item Stellt Werkzeuge zum Testen und zur Dokumentation von OpenAPI-beschriebenen \texttt{APIs} zur Verfügung
		\item Swagger Editor: Erstellen/Verändern von \texttt{APIs}
		\item Swagger Codegen: Erstellen von Software-Stubs
		\item Swagger UI: Automatisches Erzeugen von Dokumentation
	\end{itemize}
	\item \textbf{Schematischer Ansatz zum Entwickeln von Web \texttt{APIs}}
	\begin{enumerate}
		\item Grundlage: Domänen-Modell, Feature-Liste, UI-Entwürfe
		\item Identifikation von Ressourcen: Jedes Domänen-Objekt ist eine potentielle Ressource
		\item Abgleich mit der \textit{Humane Service Registry} zum Wiederverwenden existierender \texttt{APIs}
		\item Implementierung der Ressourcen-CRUD-Methoden mittels \texttt{HTTP}-Methoden
		\item Implementierung von URL-Attributen für Sortierungen, Filter, Paging etc.
	\end{enumerate}
\end{itemize}


\subsection{API Management}
\begin{itemize}
	\item Motivation: Zentrales Verwalten, Dokumentieren, Weiterentwickeln, Wiederverwenden von \texttt{APIs}
	\item \textbf{Service Registry}
	\begin{itemize}
		\item Zentrale Verwaltungskomponente; \textit{Service Provider} publisht \texttt{APIs}, \textit{Service Consumer} sucht \texttt{APIs}
		\item Beispiel: \textit{Humane Service Registry} (HSR) aller \texttt{APIs}, die bei C\&M entwickelt worden sind. Attribute sind u.a. Name, Beschreibung, Service-Gruppe und Ansprechpartner/Verantwortlicher
	\end{itemize}
	\item Bestandteile einer \texttt{API}-Managementumgebung: Management, Administrationsportal, Entwicklerportal, \texttt{API}-Microgateway
	\item \texttt{API}-Lifecycle: Design, Secure, Deploy, Publish, Analyze, Operate
	\item \textbf{\texttt{API}-Proxy}
	\begin{itemize}
		\item Verbindet \texttt{API} und Anwendung
		\item Policies: Sicherheit, Traffic-Management, Vermittlung, Erweiterungen
		\item Anwedungsfälle: JSON-Threat-Protection, Quota, Caching, Frontend einer Microservice-Umgebung, Anfragenlimitierung (Consumer muss ggf. bekannt sein)
		\item Kann Authentifizierung per (externem) \texttt{OAuth2.0}-Server vornehmen
	\end{itemize}
	\item {API}-Gateway: Serverseitiger Layer zum Umbauen, Aggregieren, Autorisieren, etc. von \texttt{API}-Aufrufen
\end{itemize}



\section{ConnectedCar}

\subsection{Einführung}
\begin{itemize}
	\item Trends im Automotive-Geschäftsbreich: Viele "`innovative"' Lösungen rund um das vernetzte Fahrzeug; technologischer Fortschritt (Hardware, Sensorik, Software, etc.); IT-Unternehmen drängen in den Markt
	\item Herausforderungen: Echtzeitereignisprocessing, Sicherheit, IAM
	\item \texttt{HERE}: Entwicklung von Schnittstellenspezifikation für ConnctedCars. Beispiele: Real Time Traffic; On-Street Parking; Road Signs; Hazard Warnings
	\item \textbf{Gefahrenwarndienst: HazardWarningServiceGroup (HazSG)}
	\begin{itemize}
		\item Warnt vor potentiellen Gefahren (Aquaplaning, Blitzeis, Verkehrsstaus, Unfälle, etc.)
		\item Dienst erhält Ereignisse vom Fahrzeug und sendet im Gegenzug Gefahreninformationen
	\end{itemize}
	\item \textbf{Complex Event Processing (CEP)}
	\begin{itemize}
		\item Definition Ereignis: "`Feingranulare, diskrete Ereignisse"' (bspw. Temperaturveränderung) versus "`grobgranulare, kompositäre Ereignise"' (Glatteiswarnung, Umleitungsempfehlung)
		\item Typisierung von Ereignissen
		\item Verarbeitung von kontinuierlichen Ereignissen in Echtzeit
		\item Mustererkennung: \texttt{[Temperatur < 0]} \(+\) \texttt{[Räder blockieren]} \(\rightarrow\) \texttt{[Blitzeis]}
		\item Esper: CEP-System für Java
		\begin{itemize}
			\item Untersuchen von Datenströmen mittels Event-Processing-Language oder Pattern Matching
			\item Erzeugen von komplexen (kompositären) Ereignistypen
			\item Dynamische Veränderung der Informationsauswertung zur Laufzeit
		\end{itemize}
		\item Aufbau eines \texttt{CEP}-Systems
		\begin{itemize}
			\item Ereignismodell: Definiert Typen und Attribute von Ereignissen
			\item Übertragungskanäle: Verschiedene Technologien möglich, i.d.R. ereignisgetriebene Subscriber-Publisher-Systeme. Beispielsweise \texttt{RabbitMQ} oder Apache Kafka
			\item Ereignisverarbeitung durch \textit{Event Processing Agents} (EPA): Diese durchsuchen den Ereignisstrom nach Mustern
		\end{itemize}
		\item Streaming Technologien
		\begin{itemize}
			\item Apache Kafka: Topic-gestützte Verteilung von Ereignisobjekten von Quellen zu Konsumenten; Zusammenfassen von Konsumenten zu \textit{ConsumerGroups}
			\item Spring Cloud Stream: Framework für ereignisgesteuerte Microservices. Abstrahiert die Streaming-Middleware (beispielsweise Apache Kafka); direkte Verbindung zwischen Anwendungen möglich (steigert die Effizienz)
		\end{itemize}
	\end{itemize}
	\item \textbf{Ergeignisverarbeitung in der ConnectedCar-Domäne}
	\begin{itemize}
		\item Im Fahrzeug: Entlastung des Backends; Echtzeitanforderungen; Hochverfügbarkeit; Datenschutz
		\item Im Backend: Kombinierbarkeit mit anderen Fahrzeugen; Änderung der Verfahren; Hinzuziehen weiterer Quellen
		\item Ereignisvorverarbeitung im Fahrzeug: Entlastung von Infrastruktur/Backend; dynamische Konfigurierbarkeit durch das Backend möglich (beispielsweise durch gezielte Anfragen)
	\end{itemize}
\end{itemize}


\subsection{Analyse und Entwurf}
\begin{itemize}
	\item Domänenmodell als Grundlage für Web-\texttt{APIs}, die für den \texttt{HazardWarningService} notwendig sind
	\item \textbf{Vorgehen}
	\begin{enumerate}
		\item Kapselung des Domänenmodells in Services
		\item Ermittlung der notwendigen Service-Operationen pro Service (statische \texttt{API}-Spezifikation)
		\item Ergänzen der dynamischen Inhalte, bei beispielsweise ablaufbeschreibende Diagramme (dynamische \texttt{API}-Spezifikation)
	\end{enumerate}
\end{itemize}



\section{Web Services with Quality}

\subsection{Softwarequalität}
\begin{itemize}
	\item \textbf{Motivation}
	\begin{itemize}
		\item Definition Qualität: Grad, zu dem Software die Erwartungen unter bestimmten Bedingungen erfüllt
		\item Wichtig um ein gegebenes Problem zu lösen; Katastrophen zu verhindern; Wartungskosten einzusparen
	\end{itemize}
	\item \textbf{Produktqualitätsmodel (ISO/IEC 25010:2011)}
	\begin{itemize}
		\item Klassifizierung von acht Haupteigenschaften, die weiter unterteilt werden
		\begin{itemize}
			\item Funktionale Eignung: Zu welchem Grad werden die Voraussetzungen erfüllt
			\item Leistungsfähigkeit
			\item Benutzbarkeit: Wirksamkeit, Effizienz, Zufriedenheit
			\item Kompatibilität: Zusammenarbeit mit anderen Produkten
			\item Zuverlässigkeit: Unter bestimmten Bedingungen für einen vorgegebenen Zeitraum
			\item Sicherheit: IAM
			\item Wartbarkeit
			\item Portierbarkeit
		\end{itemize}
	\end{itemize}
	\item \textbf{Qualitätsmetamodell}
	\begin{itemize}
		\item \textit{Quality Characteristic}: Kategorie von Qualitätsattributen (beispielsweise Wartbarkeit); kann in mehrere \textit{Quality sub-characteristics} unterteilt werden
		\item \textit{Quality Attribute}: \texttt{[1..*]} quantifizieren eine \textit{Quality Characteristic}
		\item \textit{Quality Criterion}: Zufriedenheitsschwelle pro \textit{Quality Attribute}
		\item \textit{Quality Indicator}: Weist auf die Umsetzung von \textit{Quality Attributes} hin; kann aus Design Pattern, Best Practices, etc. abgeleitet werden
		\item \textit{Quality Metric}: Formalisiert \textit{Quality Indicators}
	\end{itemize}
\end{itemize}


\subsection{Qualitätsindikatoren für Webservices}
\begin{itemize}
	\item \textbf{Best Pratices für RESTful Webservices}
	\begin{itemize}
		\item Fokus auf Wartbarkeit/Benutzbarkeit
		\item Versionierung
		\begin{itemize}
			\item Möglichkeiten: Eingebettet in die URL; Verwendung des \texttt{HTTP}-Headers
			\item RESTful Webservices müssen nicht versioniert werden (wegen des Hypermedia-Gedankens)
		\end{itemize}
		\item Beschreiben von Ressourcen
		\begin{itemize}
			\item Verwendung von domänenspezifischen Substantiven, beispielsweise in Anlehnung an die \texttt{Javascript}-Nameconventions
			\item Vermeiden von unnötigen Bezeichnungen/Ressourcen
			\item Vermeiden Plural-Singular-Mischformen
		\end{itemize}
		\item Fehlerbehandlung
		\begin{itemize}
			\item Fehlermeldungen müssen klar und verständlich sein
			\item Spezifische Nutzung von \texttt{HTTP}-Fehlercodes ("`Antwort-Code"')
			\item Inhalt einer Fehlermeldung: Separate Meldung für Entwickler und Benutzer; anwendungsspezifischer Fehlercode ("`Fehler-Code"'); Link zu weiteren Informationen
		\end{itemize}
	\end{itemize}
\end{itemize}



\section{API-Spezifikationen}

\subsection{Einführung}
\begin{itemize}
	\item Motivation: Wichtiges Unternehmensgut; kann nicht mehr ohne Weiteres geändert werden, wenn veröffentlicht (ggf. hohe Supportkosten)
	\item \textbf{Qualitätsmerkmale}
	\begin{itemize}
		\item Kundensicht: Einfach zu verstehen/nutzen
		\item Unabhängig von Technologien/Anwendungsfällen (\texttt{API}-First)
		\item Konsistent und erweiterbar (iterative Entwicklung mit Experten)
	\end{itemize}
	\item Unterscheidung zwischen Backend-Service-APIs (kapseln Geschäftslogik) und Backend-For-Frontend-APIs (kapseln mehrere Backend-Service-APIs: Feature-orientiert, UI-orientiert)
	\item Hypermedia-Aspekt wird meist ignoriert: Pfade werden "`hard-codiert"'
\end{itemize}

\subsection{Entwurf von Backend-Service-\texttt{APIs}}
\begin{enumerate}
	\item Analysieren der bestehenden Analyse-und Entwicklungsartefakte
	\item Identifikation von Ressourcen
	\item Ressourcen in \textit{Humane Service Registry} nachschauen und ggf. ergänzen
\end{enumerate}



\section{Scrum}

\subsection{Einführung}
\begin{itemize}
	\item Komplexere Anforderungen/Erwartungen erfordern moderne Methoden in der Softwareentwicklung
	\item Eigenschaften agiler Methoden: Iterativ; inkrementell; mehr Kundenwert in kürzerer Zeit; basieren auf dem \textit{Agilen Manifest (2001)}
	\item Beispiele: Extrem Programming (XP); Kanban
	\item \textbf{Scrum}
	\begin{itemize}
		\item Ziel: Geregelte Zusammenarbeit in selbstorganisierenden Teams
		\item Komplexitätsreduktion durch Transparenz; Inspektion; Adaption
		\item Sprints: Regelmäßiges, kurzes Intervall (2-3 Wochen), in der UserStories umgesetzt werden. Ergebnis ist ein lauffähiges Produktinkrement (vollständig umgesetzt und getestet)
	\end{itemize}
\end{itemize}


\subsection{Rollen}
\begin{itemize}
	\item ProductOwner: Verantwortlich für Rentabilität; Produktvision; Überwachen der Qualität; Produktpflege; Releases. Allerdings kein Projektleiter
	\item ScrumMaster: Teamcoach, der das Entwicklungsteam extern vertritt und für das Gelingen verantwortlich ist. Verwaltet das \textit{Impediment Backlog}
	\item Entwicklungsteam: Selbstorganisierte Umsetzung der Softwareanforderungen in Teams von 7 \(\pm\) 2 Personen; Verantwortlich für Qualität. Auch kein Projektleiter
\end{itemize}


\subsection{Artefakte}
\begin{itemize}
	\item \textbf{Product Backlog}
	\begin{itemize}
		\item Priorisierte Anforderungsliste in Form von UserStories
		\item ProductOwner sortiert die Liste nach Geschäftswert
		\item Meist: Aufsplitten der UserStories beim Wandern "`nach oben"'
		\item \texttt{DEEP}-Kriterien für ein gutes Product Backlog
		\begin{description}
			\item[Details:] Hinreichend detailliert
			\item[Estimated:] Geschäftswert schätzbar
			\item[Emergent:] Liste wird dynamisch verändert/aktualisiert
			\item[Prioritized:] Nach Priorisierung sortiert
		\end{description}
		\item Überarbeitung der Sortierung in \textit{Grooming-Meetings} (kein offizieller Bestandteil von Scum)
	\end{itemize}
	\item \textbf{User Stories}
	\begin{itemize}
		\item Feature aus User-Sicht in ein bis zwei kurzen Sätzen: Wer?, Was?, Warum?
		\item Bestandteile: Id, Aufwand in \textit{StoryPoints}, Akzeptanzkriterien
	\end{itemize}
	\item \textbf{Planning Poker}
	\begin{itemize}
		\item Aufwandschätzung einer UserStory; kein spezifische Methode von Scrum vorgschrieben, \textit{Planning Poker} wird häufig eingesetzt
		\item Alle Teammitglieder geben unabhängig Schätzungen ab; Extrema müssen ihren Standpunkt erklären; wiederholen bis Konsens gefunden
	\end{itemize}
	\item \textbf{Sprint Backlog}
	\begin{itemize}
		\item Enthält: Alle UserStories des aktuellen Sprints sowie die abgeleiteten Aufgaben
		\item Visualisiert durch (virtuelles) Taskboard \(\rightarrow\) schafft Transparenz
	\end{itemize}
	\item Impediment Backlog: ScrumMaster-geführtes Backlog, in dem alle Probleme, die während des DailyScrum auftreten, gelistet werden
	\item \textbf{Charts}
	\begin{description}
		\item[Burndown-Chart:] Zeigt den Restaufwand innerhalb eines Sprints
		\item[Velocity-Chart:] Zeigt die Leistungsfähigkeit des Teams in StoryPoints pro Sprint
	\end{description}
\end{itemize}


\subsection{Meetings}
\begin{itemize}
	\item \textbf{Sprint Planning I (Was?)}
	\begin{itemize}
		\item Formulierung des Sprint-Ziels sowie Auswahl der UserStories (nach "`Bauchgefühl"') \(\rightarrow\) Erstellen des \textit{Sprint-Backlogs}
		\item Festlegen der Akzeptanzkriterien
		\item Detailliertes Verständnis für den Kundenwunsch nötig
	\end{itemize}
	\item \textbf{Sprint Planning II (Wie?)}
	\begin{itemize}
		\item Ziel: Design und Entwurf für das Produktinkrement
		\item Folgt am gleichen Tag
		\item Ableiten von Aufgaben aus den UserStories sowie Abschätzen des Zeitaufwands
	\end{itemize}
	\item \textbf{Daily Scrum}
	\begin{itemize}
		\item Tägliches Meeting zur Synchronisation
		\item Nicht länger als 15 min; idealerweise stehend; keine Problemlösungen
		\item ScrumMaster moderiert und achtet auf Regeleinhaltung
	\end{itemize}
	\item \textbf{Sprint Review}
	\begin{itemize}
		\item Ziel: Abschluss des Sprints, meist mit Kunde
		\item Analyse des Inkrements unter Einhaltung der Abnahmekriterien
		\item Feedbacksammlung zur Produktverbesserung
	\end{itemize}
	\item Sprint Retrospektive: Optimierung der Zusammenarbeit und der Entwicklungsprozesse ohne den Kunden
\end{itemize}


\subsection{Agile Entwicklung mit Scrum}
\begin{itemize}
	\item Managementwerkzeug ohne Vorgabe von Entwicklungspraktiken \(\rightarrow\) Team entscheidet darüber eigenverantwortlich (Vermeidung von "`technische Schulden"')
	\item \textbf{Architekturvision}
	\begin{itemize}
		\item Gemeinsames Verständnis inklusive grober, gemeinsamer Architekturvision
		\item CAS-Eigenschaften einer Architekturvision
		\begin{description}
			\item[Clear:] Darstellung der Umsetzung von nicht-funktionalen Anforderungen; Einsatz einer Modellierungssprache
			\item[Accepted:] Von allen mitgetragen; keine architektonischen Alleingänge
			\item[Short:] Kein vollständiges Modell
		\end{description}
	\end{itemize}
	\item Collective Ownership: Gemeinsame Verantwortung des Teams für den gesammten Quellcode
	\item Pair Programming: Zwei Entwickler arbeiten zusammen an einem Computer; periodischer Rollenwechsel nach mehreren Minuten
	\item Refactoring: Code-Updates ohne das externe Verhalten zu ändern; lauffähige Testfälle wichtig
	\item \textbf{Inkrementeller Entwurf}
	\begin{itemize}
		\item Entwicklung erfolgt in kleinen Teilen \(\rightarrow\) flache Aufwandskurve für neue Features \(\rightarrow\) Code muss leicht änderbar/wartbar sein
		\item Einhaltung der SOLID-Entwurfskriterien
	\end{itemize}
	\item Testgetriebene Entwicklung: Testfälle werden vor der eigentlichen Entwicklung geschrieben
	\item \textbf{Verhaltensgetriebene Entwicklung}
	\begin{itemize}
		\item Ziel: Automatisierte Überprüfung der Anforderungen
		\item Während der Anforderungserhebung in "`natürlicher"' Sprache definiert (beispielsweise \texttt{Gherkin})
		\item Erleichtert den Übergang von Anforderungssprache zu Implementierungssprache
	\end{itemize}
	\item Continuous Integration
\end{itemize}



\section{Sicherheit}

\subsection{Sicherheitsorientierter Entwicklungsprozess}
\begin{itemize}
	\item Informationssicherheit: Schutz von Informationen, insbesondere der Systeme, die sie benutzen, speichern und übermitteln
	\item Sicherheit wird meist nur zweitrangig betrachtet. Historische Systeme waren lange isoliert, erst durch das Internet stieg das Schutzbedürfnis
	\item \textbf{Schutzziele}
	\begin{itemize}
		\item CIA: Confidentiality (Vertraulichkeit), Integrity (Integrität) und Availability (Verfügbarkeit)
		\item Viele weitere Schutzziele möglich, beispielsweise Authentizität, Nachweisbarkeit, Verantwortlichkeit, Datenschutz
	\end{itemize}
	\item \textbf{Identity and Access Management (IAM)}
	\begin{itemize}
		\item Ziel: Schutz der Integrität. \textit{Wer} darf \textit{welche} Ressourcen \textit{wie} nutzen?
		\item IAM regelt den Ressourcenzugriff
		\item Trennung in Enterprise und Consumer IAM
	\end{itemize}
\end{itemize}


\subsection{Authentifizierung und Autorisierung in Web-Anwendungen}
\begin{itemize}
	\item \textbf{Ablauf}
	\begin{enumerate}
		\item Benutzer sendet Authentifizierungsdaten an die Web-Anwendung mit Referenzmonitor
		\item Web-Anwendung authentifiziert Benutzer am Benutzerverzeichnis, dieses erzeugt ein Token, das an den Benutzer übergeben wird
		\item Benutzer sendet das Token an die Web-Anwendung. Diese prüft die dazugehörigen Berechtigungen über den Autorisierungsserver
		\item Autorisierungsserver validiert das Token am Benutzerverzeichnis
		\item Autorisierungsserver antwortet der Web-Anwendung
	\end{enumerate}
	\item Absicherung traditioneller, monolithischer Web-Anwendungen: Abfangen von Anfragen durch zentrale Komponente (beispielweise Referenzmonitor)
	\item Probleme bei MicroServices: Zugriffskontrolle muss in jedem MicroService implementiert werden; Anfragen durch Weitergabe der Anmeldedaten des Benutzers
	\item \textbf{Ablauf beim Servicezugriff}
	\begin{itemize}
		\item Im Namen des Benutzers
		\item Im eigenen Namen
	\end{itemize}
	\item \textbf{Tokens}
	\begin{itemize}
		\item Idee: Token als Autorisierungsnachweis bei Anfragen
		\item Typen
		\begin{itemize}
			\item By-Reference-Token: Beinhaltet keine zusätzlichen Informationen außerhalb des Netzwerks
			\item By-Value-Token: Beinhaltet alle notwendigen Informationen
		\end{itemize}
		\item Profile
		\begin{itemize}
			\item Bearer-Token: Kann von jedem benutzt werden (vgl. Bargeld)
			\item Holder-of-Key-Token: Kann nur durch authentifizierten Benutzer genutzt werden (vgl. Krditkarte)
		\end{itemize}
	\end{itemize}
\end{itemize}


\subsection{Einsatz von Sicherheitsmustern beim Systementwurf}
\begin{itemize}
	\item Sicherheitsmuster: Strukturierte Beschreibung von Sicherheitswissen; bekannte und erprobte Lösungen für Sicherheitsprobleme auf unterschiedlichen Abstraktionsebenen
	\item Sicherheitsanforderungen lassen sich auf Schutzziele zurückführen; Sicherheitsmuster zur Erfüllung von Sicherheitsanforderungen verfügbar
	\item \textbf{Zugriffskontrolle}
	\begin{itemize}
		\item Referenzmonitor
		\begin{itemize}
			\item Fängt alle Ressourcenanfragen ab und prüft, ob diese autorisiert sind und leitet im Erfolgsfall an das geschützte Objekt weiter
			\item Sehr allgemeingehalten; keine Vorgaben hinsichtlich Authentifizierung und Autorisierung
		\end{itemize}
		\item Sicherheitsmuster für Authentifizierung: Wissen, Besitz, Sein
		\item Sicherheitsmuster für Autorisierung: Attributbasiert, rollenbasiert, identitätsbasiert, etc.
	\end{itemize}
	\item \textbf{Beispiele}
	\begin{itemize}
		\item Referenzmonitor
		\item Identitätsbasierte Zugriffskontrolle: Zugriffsmatrix in der Form \(\lbrack Subjekt, Ressource \rbrack \longrightarrow Recht\)
		\item Rollenbasierte Zugriffskontrolle
		\begin{itemize}
			\item Rollen: Rechte/Pflichten, die zusammengefasst einer Tätigkeit zugeordnet werden
			\item Werden Personen oder Gruppen zugeordnet
			\item Meist nur auf einzelne Systeme beschränkt
		\end{itemize}
		\item Attributsbasierte Zugriffskontrolle: An boolesche Algebra angelehnt; Nutzung von Umweltinformationen möglich
		\item Authentifizierung: Prüfung einer digitalen Identität; Auswahl des Verfahrens wird durch den Schutzbedarf bestimmt; Mehr-Faktor-Authentifizierung möglich
		\item Authentifizierungsmethoden im \textit{Consumer IAM}
		\begin{itemize}
			\item Über soziales Netzwerk
			\item Risikobasiert: Authentzitätswahrscheinlichkeit auf Basis gesammelter Daten berechnen
			\item QR-Codes
			\item Mobile Anwendung: Zusenden von Anmeldelinks
			\item Videokonferenz mit Personalausweis
		\end{itemize}
	\end{itemize}
\end{itemize}
